\documentclass[a5paper, 10pt]{article}

% Текст
\usepackage[utf8]{inputenc} % UTF-8 кодировка
\usepackage[russian]{babel} % Русский язык
\usepackage{indentfirst} % красная строка в первом параграфе в главе
% Отображение страниц
\usepackage{geometry} % размеры листа и отступов
\geometry{
	left=12mm,
	top=25mm,
	right=15mm,
	bottom=17mm,
	marginparsep=0mm,
	marginparwidth=0mm,
	headheight=10mm,
	headsep=7mm,
	nofoot}
\usepackage{afterpage,fancyhdr} % настройка колонтитулов
\pagestyle{fancy}
\fancypagestyle{style}{ % создание нового стиля style
	\fancyhf{} % очистка колонтитулов
	\fancyhead[LO, RE]{Дифференциальные уравнения} % название документа наверху
	\fancyhead[RO, LE]{\leftmark} % название section наверху
	\fancyfoot[RO, LE]{\thepage} % номер страницы справа внизу на нечетных и слева внизу на четных
	\renewcommand{\headrulewidth}{0.25pt} % толщина линии сверху
	\renewcommand{\footrulewidth}{0pt} % толцина линии снизу
}
\fancypagestyle{plain}{ % создание нового стиля plain -- полностью пустого
	\fancyhf{}
	\renewcommand{\headrulewidth}{0pt}
}
\fancypagestyle{title}{ % создание нового стиля title -- для титульной страницы
	\fancyhf{}
	\fancyhead[C]{{\footnotesize
			Министерство образования и науки Российской Федерации\\
			Федеральное государственное автономное образовательное учреждение высшего образования
	}}
	\fancyfoot[C]{{\large 
			Санкт-Петербург, 2022-2023
	}}
	\renewcommand{\headrulewidth}{0pt}
}

% Математика
\usepackage{amsmath, amsfonts, amssymb, amsthm} % Набор пакетов для математических текстов
\usepackage{dmvnbase} % мехматовский пакет latex-сокращений
\usepackage{cancel} % зачеркивание для сокращений
% Рисунки и фигуры
\usepackage[pdftex]{graphicx} % вставка рисунков
\usepackage{wrapfig} % вставка фигур, обтекая текст
\usepackage{subcaption} % расположение нескольких объектов внутри одной фигуры
\usepackage{caption} % для настройки подписей
\captionsetup{figurewithin=none,labelsep=period, font={small,it}} % настройка подписей к рисункам
% Рисование
\usepackage{tikz} % рисование
\usepackage{pgfplots} % графики
% Таблицы
\usepackage{multirow} % объединение строк
\usepackage{multicol} % объединение столбцов
% Остальное
\usepackage[unicode, pdftex]{hyperref} % гиперссылки
\usepackage{enumitem} % нормальное оформление списков
\setlist{itemsep=0.15cm,topsep=0.15cm,parsep=1pt} % настройки списков
% Теоремы, леммы, определения...
\theoremstyle{definition}
\newtheorem{Def}{Определение}
\newtheorem*{Axiom}{Аксиома}
\theoremstyle{plain}
\newtheorem{Th}{Теорема}
\newtheorem{Lem}{Лемма}
\newtheorem{Cor}{Следствие}
\newtheorem*{Prop}{Предложение}
\newtheorem{Ex}{Пример}
\theoremstyle{remark}
\newtheorem*{Note}{Замечание}
\newtheorem*{Solution}{Решение}
\newtheorem*{Proof}{Доказательство}
% Свои команды
\newcommand{\comb}[1]{\left[\hspace{-4pt}\begin{array}{l}#1\end{array}\right.\hspace{-5pt} } % совокупность уравнений
% Титульный лист
\newcommand*{\titlePage}{
	\thispagestyle{title}
	\begingroup
	\begin{center}
%		{\footnotesize
%			Министерство образования и науки Российской Федерации\\
%			Федеральное государственное автономное образовательное учреждение высшего образования
%		}
%		
		\vspace*{6ex}
		
		{\small
			САНКТ-ПЕТЕРБУРГСКИЙ НАЦИОНАЛЬНЫЙ ИССЛЕДОВАТЕЛЬСКИЙ УНИВЕРСИТЕТ ИНФОРМАЦИОННЫХ ТЕХНОЛОГИЙ, МЕХАНИКИ И ОПТИКИ	
		}
		
		\vspace*{2ex}
		
		{\normalsize
			Факультет систем управления и робототехники
		}
		
		\vspace*{15ex}
		
		{\Large \bfseries 
			Дифференциальные уравнения
		}
	\end{center}
	\vspace*{20ex}
	\begin{flushright}
		{\large 
			\underline{Выполнил}: студент гр. \textbf{R32353}\\
			\begin{flushright}
				\textbf{Магазенков Е. Н.}\\
			\end{flushright}
		}
		
		\vspace*{5ex}
		
		{\large 
			\underline{Преподаватель}: \textit{Бабушкин М. В.}
		}
	\end{flushright}	
	\newpage
	\setcounter{page}{1}
	\endgroup}
	


\begin{document}
	\titlePage
	\pagestyle{style}
	\part{}
	\section[ЛУ 1-ого порядка]{Линейные уравнения 1-ого порядка}
	\begin{Def}
		Дифференциальное уравнение вида 
		\begin{equation}
			y' = p(x) y + q(x),
		\end{equation}
		называют линейным неоднородным дифференциальным уравнением первого порядка (\textsc{ЛНУ} или просто \textsc{ЛУ}).
	\end{Def}
	
	\begin{Def}
		Дифференциальное уравнение вида 
		\begin{equation}
			y' = p(x) y
		\end{equation}
		называют линейным однородным дифференциальным уравнением первого порядка (\textsc{ЛОУ}).
	\end{Def}
	\begin{Note}
		Вообще, мы уже неоднократно сталкивались и решали уравнения такого вида. Однако для строгого обоснования наших решений рассмотрим следующую лемму.
	\end{Note}
	\begin{Lem}[Общее решение ЛОУ]
		Пусть в линейном однородном уравнении $y' = p(x) y$ функция $p(x) \in C\hr{a,b}$.
		
		Тогда его общее решение имеет вид \begin{equation}
			y = c\cdot e^{\int\!p},
		\end{equation}
		где $c\in \R$ -- произвольная константа и под $\int \!p$ понимается какая-то производная функции $p(x)$.
		\begin{Proof}
			Воспользуемся эквивалентным преобразованием
			\[
			y'=p(x)y \Leftrightarrow dy = p(x) y dx.
			\]
			Рассмотрим несколько возможных случаев:
			\begin{enumerate}
				\item $y=0$ -- очевидно решение,
				\item при $y>0$: разделим на $y$ с обеих сторон
				\[
				\frac{dy}{y} = p(x) dx.
				\]
				Проинтегрируем обе части:
				\[
				\int \frac{dy}{y} = \int p(x) dx,
				\]				
				\[
				\ln y = \int p(x) dx,
				\]
				\[
				y = A\cdot e^{\int\!p}, \quad\text{где } A>0.
				\]
				\item при $y<0$: аналогично, но появляется минус, который можно засунуть в константу.
				\[
				y = B\cdot e^{\int\!p}, \quad\text{где } B<0.
				\]
				\item Осталось разобрать случай, когда интегральная кривая проходит через границу $y=0$. Однако, рассматривая все кривые, видно, что они заданы строго в одной полуплоскости относительно $y=0$:
				\[
				y = A\cdot e^{\int\!p} > 0, \quad\text{где } A>0, \quad y = B\cdot e^{\int\!p} < 0, \quad\text{где } B<0.
				\]
			\end{enumerate}
			Таким образом, действительно, общее решение ЛОУ можно записать в виде $	y = c\cdot e^{\int\!p}$.
		\end{Proof}
	\end{Lem}
	\begin{Note}
		Теперь мы строго доказали, ранее использовавшиеся факты. Как вывод из этого, мы получаем, что теперь можно каждый раз не решать ЛОУ, а просто пользоваться формулой. Или хотя бы всегда проверять, похоже ли решение на полученное в общем виде.
	\end{Note}
	\begin{Note}
		Далее мы будем рассматривать общее решение неоднородного уравнения. Мы используем достаточно интересный метод доказательства: так, мы предоставим какое-то решение, которое мы назовем общим, а далее докажем, что любое другое решение на самом деле задается именно нашим выражением. 
		
		Оказывается, что такой метод можно применять и для решения любых уравнений. Достаточно лишь показать, что представленное выражение является решением, а также, что любое другое произвольное решение задается этим выражением.
	\end{Note}
	\begin{Lem}[Общее решение ЛНУ]
		Пусть в линейном уравнении $y' = p(x) y + q(x)$ функции $p(x),\, q(x) \in C\hr{a,b}$.
		
		Тогда его общее решение имеет вид \begin{equation}
			y = \hr{\int q \cdot e^{-\int\!p} + c} \cdot e^{\int \! p},
		\end{equation}
		где $c\in \R$ -- произвольная константа и под $\int \!f$ понимается какая-то производная функции $f(x)$.
		\begin{Proof}
			\begin{itemize}
				\item Докажем, что данное данное множество решений включено в общее множество решений исходного уравнения. Проще говоря, проверим, правда ли, что представленное выражение является решением.
				
				Найдем $y'(x)$:
				\[
				y' = q\cdot e^{-\int\! p} \cdot e^{\int\!p} + \hr{\int q\cdot e^{-\int\!p} + c} \cdot p e^{\int\!p} = q + \hr{\int q\cdot e^{-\int\!p} + c} \cdot p e^{\int\!p}.
				\]
				Тогда, подставляя в исходное уравнение:
				\[
				q + \hr{\int q\cdot e^{-\int\!p} + c} \cdot p e^{\int\!p} \equiv p \cdot \hr{\int q \cdot e^{-\int\!p} + c} \cdot e^{\int \! p} + q,
				\]
				получаем верное тождество.
				\item Докажем, что произвольное решение задается формулой $y = \hr{\int q \cdot e^{-\int\!p} + c} \cdot e^{\int \! p}$.
				
				Пусть $\varphi \in \hr{\alpha,\beta}$ -- решение, не задающееся этой формулой.
				
				Рассмотрим $x_0 \in \hr{\alpha, \beta} : \; \varphi(x_0) = y_0$.
				
				Найдем такое $c \in \R$, что наше решение проходит через точку $\hr{x_0, y_0}$. 
				\[
				\left. \hr{\int q \cdot e^{-\int\!p} + c} \cdot e^{\int \! p} \right|_{x=x_0} = y_0,
				\]
				\[
				c = \left. \hr{y_0 \cdot e^{-\int\!p} - \int q\cdot e^{-\int\!p}} \right|_{x=x_0}.
				\]
				Пусть решение с этим $c$ -- решение $\psi$. Тогда мы получили два решения задачи Коши $y = \hr{\int q \cdot e^{-\int\!p} + c} \cdot e^{\int \! p}$ с начальными условиями $y(x_0) = y_0$ на интервале $\hr{\alpha,\beta}$. Так как $f(x, y) = p(x) y+q(x) \in C\hr(a,b)$, то по теореме об единственности решения задачи Коши $\varphi = \psi$, что противоречит предположению о том, что $\varphi$ не задается решением вида $	y = \hr{\int q \cdot e^{-\int\!p} + c} \cdot e^{\int \! p}$. 
				
				Таким образом, действительно любое решение можно представить в виде $y = \hr{\int q \cdot e^{-\int\!p} + c} \cdot e^{\int \! p}$.
			\end{itemize}
			Объединяя эти два факта, мы получаем, что показанное нами решение действительно является общим решением линейного дифференциального уравнения.
		\end{Proof}
	\end{Lem}
	
	\begin{Note}
		В итоге мы имеем формулу для решения ЛУ, в которую можно подставить нужные значения. Однако достаточно тяжело помнить ее наизусть, поэтому нужно иметь какой-то метод, который сможет нас привести к этому решению. Напомним, что любой метод, который будет давать решение вида $	y = \hr{\int q \cdot e^{-\int\!p} + c} \cdot e^{\int \! p}$ окажется верным, так как мы уже доказали, что это общее решение. 
	\end{Note}

	\begin{Prop}[Метод Лагранжа или метод вариации произвольной постоянной]
		Пусть стоит задача решить линейное уравнение $y'=p(x)y+q(x)$.
		
		Рассмотрим соответствующее однородное уравнение, то есть уравнение $y'=p(x)y$. Его общее решение мы знаем (либо можем найти): $y=c\cdot e^{\int\!p}$. 
		
		Рассмотрим теперь $c$ не как константу, а как функцию $c(x)$.
		
		Подставляя $y=c(x) \cdot e^{\int\!p}$ в исходное линейное уравнение, получаем:
		\[
		c'(x) \cdot e^{\int\!p} + \cancel{c(x) \cdot p e^{\int\!p}} = \cancel{p c(x) \cdot e^{\int\!p}} + q,
		\]
		\[
		c'(x) = q\cdot e^{-\int\!p}.
		\]
		Это уравнение мы снова можем решить:
		\[
		c(x) = \int q\cdot e^{-\int\!p} + \tilde{c}.
		\]
		Тогда, возвращаясь к решению однородного уравнения и подставляя $c$ туда, получаем
		\[
		y = \hr{\int q\cdot e^{-\int\!p} + \tilde{c}}\cdot e^{\int\!p},
		\]
		то есть общее решение линейного уравнения.
	\end{Prop}

	\section[Бернулли и Рикатти]{Уравнения Бернулли и Рикатти}
	\begin{Note}
		Существует огромное количество уравнений первого порядка, которые можно свести к линейному какой-либо заменой. В этом пункте будут разобраны такие уравнения, представленные в XVII веке Якобом Бернулли \footnote{Якоб Бернулли (1655--1705, Швейцария)} и Рикатти \footnote{Рикк\'{а}ти Якопо Франческо (1676--1754, Италия)}.
	\end{Note}
	
	\begin{Def}
		Уравнение вида 
		\begin{equation}
			y' = p(x) y + q(x) y^{\alpha},
		\end{equation}
		где $\alpha\neq \hc{0,1}$, называется дифференциальным уравнением Бернулли.
	\end{Def}
	
	\begin{Lem}
		Уравнение Бернулли $y' = p(x) y + q(x) y^{\alpha}$ при $y\neq0$ заменой $t = y^{1-\alpha}$ сводится к линейному дифференциальному уравнению.
		\begin{Proof}
			Рассмотрим уравнение Бернулли 
			\[
			y' = p(x) y + q(x) y^{\alpha}.
			\]
			Поделим обе части уравнения на $y^{\alpha}$:
			\[
			\frac{y'}{y^\alpha} = p(x) y^{1-\alpha} + q(x).
			\]
			Сделаем замену $t = y^{1-\alpha}$, тогда $t' = (1-\alpha)\frac{y'}{y^{\alpha}}$:
			\[
			\frac{1}{1-\alpha} t' = p(x) t + q(x),
			\]
			\[
			t' = \hr{1-\alpha}\hr{p(x) t + q(x)}.
			\]
			Получили линейное дифференциальное уравнение.
		\end{Proof}
	\end{Lem}
	
	\begin{Def}
		Уравнение вида 
		\begin{equation}
			y' = p(x) y^2 + q(x) y + r(x)
		\end{equation}
		называется дифференциальным уравнением Бернулли.
	\end{Def}
	\begin{Note}
		Уравнение Бернулли является частным случаем уравнения Рикатти при $r(x)\equiv 0$. Рикатти был знаком с семьей Бернулли, поэтому связь между этими уравнениями неслучайна.
	\end{Note}
	
	\begin{Lem}
		Пусть $\varphi$ -- какое-то решение уравнения Рикатти $y' = p(x) y^2 + q(x) y + r(x)$. Подстановка $y = z + \varphi$ сводит это уравнение к уравнению Бернулли.
		\begin{Proof}
			Найдем $y'$:
			\[
			y' = z' + \varphi'.
			\]
			Так как $\varphi$ -- решение уравнения Рикатти, то 
			\[
			\varphi' = p(x) \varphi^2 + q(x) \varphi + r(x).
			\]
			Тогда $y' = z' + p(x) \varphi^2 + q(x) \varphi + r(x)$.
			
			Подставим замену в уравнение Рикатти
			\[
				z' + p(x) \varphi^2 + q(x) \varphi + \cancel{r(x)} = p(x) (z+\varphi)^2 + q(x) (z+\varphi) + \cancel{r(x)},
			\]
			\[
				z' = z \underbrace{\hr{2p(x)\varphi + q(x)}}_{P(x)} + \underbrace{p(x)}_{Q(x)} z^2,
			\]
			\[
				z' = P(x) z + Q(x) z^2.
			\]
			Получили уравнение Бернулли для $\alpha=2$.
		\end{Proof}
	\end{Lem}
	
	\section[УПД]{Уравнение в полных дифференциалах}
	\begin{Def}
		Пусть существует функция $u$ такая, что $du = P(x,y)dx + Q(x,y)dy$ (то есть $u_x'=P$, $u_y'=Q$). Тогда уравнение вида 
		\begin{equation}
			P(x,y)dx + Q(x,y)dy = 0
		\end{equation}
		называется уравнением в полных дифференциалах (\textsc{УПД}).
	\end{Def}
	
	\begin{Th}[Общее решение УПД]
		Пусть $u \in C^1 (G)$, причем $u_x'=P$, $u_y'=Q$. Тогда функция $\varphi(x)$ является общим решением уравнения $P(x,y)dx + Q(x,y)dy = 0$ на интервале $\hr{\alpha, \beta}$ в том и только том случае, когда $\varphi \in C^{1} \hr{\alpha,\beta}$ и $\varphi$ неявно задается уравнением $u(x,y) = c$ при некотором $c\in \R$.
		\begin{Proof}
			\underline{Необходимость}:\\
			Так как $\varphi$ -- решение, то $\varphi \in C^{1} \hr{\alpha,\beta}$ и 
			\[
			P(x,\varphi(x))dx + Q(x,\varphi(x))d(\varphi(x)) \equiv 0,
			\]
			то есть 
			\[
			P(x,\varphi(x)) + Q(x,\varphi(x))\varphi'(x) \equiv 0.
			\]
			Заметим, что левая часть есть полная производная функции $u(x, \varphi(x))$.
			\[
			\frac{d}{dx} u(x,\varphi(x)) \equiv 0,
			\]
			\[
			u(x, \varphi(x)) \equiv c.
			\]
			Таким образом, получаем, что $\varphi$ действительно неявно задана уравнением $u(x,y)=c$.
			
			\underline{Достаточность}:\\
			Так как $\varphi$ неявно задана уравнением $u(x,y)=c$, то 
			\[
			u(x, \varphi(x)) \equiv c.
			\]
			Тогда \[
			\frac{d}{dx} u(x,\varphi(x)) \equiv 0,
			\]
			то есть 
			\[
			P(x,\varphi(x)) + Q(x,\varphi(x))\varphi'(x) \equiv 0.
			\]
			Так как $\varphi \in C^{1} \hr{\alpha,\beta}$, то $\varphi$ является решением уравнения $P(x,y)dx + Q(x,y)dy = 0$ по определению решения дифференциального уравнения.
		\end{Proof}
	\end{Th}

	\begin{Note}
	Назревает хороший вопрос: откуда эту функцию $u$ взять и почему вообще она существует?
	
	Пусть есть функция $u: u_x'=P, \; u_y' = Q, \; u\in C^2(G)$. Рассмотрим вторые производные:
	\[
	\left.
	\begin{split}
		u_xy''=P_y', \\
		u_yx'' = Q_x'
	\end{split}
	\right\} \implies P_y' = Q_x'.
	\]
	На основе этого утверждения появляется следующая теорема.
	\end{Note}
	
	\begin{Th}[Признак УПД]
		Пусть $P, Q \in C^1(G)$, причем $P_y' = Q_x'$, где $G$ -- односвязная область.
		Тогда существует функция $u : u_x' = P, u_y' = Q$. Кроме того, все такие функции имеют вид 
		\[
		u(\tilde{x}, \tilde{y}) = \int\limits_{\gamma(\tilde{x}, \tilde{y})} P(x,y)dx + Q(x,y)dy + c,
		\]
		где $c \in \R$, $\gamma(\tilde{x}, \tilde{y})$ -- кривая в области $G$, соединяющая точки $\hr{x_0,y_0}$ и $\hr{\tilde{x}, \tilde{y}}$.
	\end{Th}
	
	\begin{Note}
		Таким образом, при определенных условиях мы поняли, как можно найти эту функцию $u$. И далее, решая уравнение $u=c$, можем найти общее решение УПД.
		
		Однако вычисление данного криволинейного интеграла зачастую является непростой задачей, поэтому рассмотрим другие методы.
	\end{Note}
	
	 \begin{Def}
		Функция $u$ при условии $u_x'=P$, $u_y'=Q$ называется потенциалом поля $\hr{P,Q}$, а поле $\hr{P,Q}$ -- потенциальным полем. 
	\end{Def}
	
	\begin{Ex}
		\[
		e^{-y}dx - \hr{xe^{-y}+2y} dy = 0.
		\]
		\begin{Solution}
			Область определения уравнения есть $\R^2$ -- односвязное множество. 
			
			Рассмотрим производные коэффициентов $P=e^{-y}$ и $Q = \hr{xe^{-y}+2y}$:
			\[
				P_y' = -e^{-y} = Q_x'.
			\]
			Таким образом, по признаку -- это уравнение в полных дифференциалах. 
			
			Не будем вычислять криволинейный интеграл. Но рассмотрим систему
			\[
			\case{u'_x = e^{-y}, \\ 
					u_y' = -xe^{-y}-2y}.
			\]
			Рассмотрим потенциал в какой-то точке с фиксированной ординатой $y_0$: 
			\[
			u_x' (x, y_0) = e^{-y_0} \implies u(x, y_0) = \int e^{-y_0} dx = xe^{-y_0} + c(y_0).
			\]
			Подставляя во второе уравнение системы, получаем
			\[
			\hr{xe^{-y} + c(y)}_y' = -xe^{-y} - 2y,
			\]
			\[
			\cancel{-xe^{-y}} + c'(y) = \cancel{-xe^{-y}} - 2y,
			\]
			\[
			c' = -2y,
			\]
			\[
			c(y) = -y^2 + A.
			\]
			Таким образом, $u(x,y) = xe^{-y} -y^2 + A$, а значит уравнение $ xe^{-y} -y^2 = C$ задает решение УПД.
		\end{Solution}
	\end{Ex}
	
	\begin{Note}
		В примере был показан другой способ решения УПД, избегающий криволинейное интегрирование. Однако данный способ далеко не всегда оказывается возможен. По крайней мере, не всегда можно взять интеграл, чтобы найти $u(x, y_0)$ (при плохой области интеграл по прямой будет достаточно сложен).
	\end{Note}
	
	\begin{Def}
		Функция $\mu(x,y) \neq 0$ называется интегрирующим множителем уравнения $P(x,y)dx + Q(x,y)dy = 0$, если при домножении этого уравнения на $\mu$ получается уравнение в полных дифференциалах.
		\[
		\mu P(x,y)dx + \mu Q(x,y)dy = 0 \text{ --- УПД.}
		\]
	\end{Def}
	
	\begin{Note}
		Если $\mu$ -- интегрирующий множитель уравнения $P(x,y)dx + Q(x,y)dy = 0$, причем $\mu, P, Q \in C^1(G)$. Тогда $\hr{\mu P}_y' = \hr{\mu Q}_x'$. Расписывая производную получаем уравнение в частных производных
		\[
		\mu_y' P + \mu P_y' = \mu_x' Q + \mu Q_x'. 
		\] 
		Решать его оказывается совсем непросто, однако чисто теоретически это является способом нахождения интегрирующего множества.
		
		Однако стоит помнить, что нам не требуется находить общее решение. Нам достаточно лишь какое-то частное решение. Иногда его можно найти, как в следующем примере.
	\end{Note}
	\begin{Ex}
		Рассмотрим линейное уравнение
		\[
		y' = p(x) y + q(x), 
		\]
		где $p\neq 0$. Попробуем найти его интегрирующий множитель.
		\begin{Solution}
			Перепишем исходное уравнение 
			\[
			\hr{p(x) y + q(x)}dx - dy = 0.
			\]
			Заметим, что это уравнение не является уравнением в полным дифференциалах.
			
			Запишем уравнение для нахождения интегрирующего множителя.
			\[
			\mu_y' \hr{p(x) y + q(x)} + \mu p(x) = -\mu_x'.
			\]
			Будем искать интегрирующий множитель в виде $\mu = \mu(x)$. Тогда первое слагаемое слева обнулится
			\[
			\mu p(x) = -\mu_x',
			\]
			\[
			\mu = C e^{-\int\!p}.
			\]
			Так как нам нужно лишь какое-то решение, то рассмотрим любое $C$, например $C=1$.
			
			Таким образом, $\mu = e^{-\int\!p}$ -- интегрирующий множитель.
			
			Умножим на $\mu$ исходное уравнение
			\[
				y' e^{-\int\!p} = p(x) y e^{-\int\!p}+ q(x)e^{-\int\!p},
			\]
			\[
				y' e^{-\int\!p} - p(x) y e^{-\int\!p} = q(x)e^{-\int\!p}.
			\]
			Заметим, что в левой части стоит производная произведения $\hr{ye^{-\int\!p}}$
			\[
			\hr{ye^{-\int\!p}}= q(x)e^{-\int\!p},
			\]
			\[
			ye^{-\int\!p} = \int q(x)e^{-\int\!p} + A,
			\]
			\[
			y = \hr{\int q(x)e^{-\int\!p} + A}e^{\int\!p}.
			\]
			Таким образом, получили общее решение линейного уравнения, а значит решение привело к верному ответу.
		\end{Solution}
	\end{Ex}
	
	\begin{Note}
		Таким образом, мы получили еще один способ нахождения решения линейного уравнения, которым можно пользоваться на практике. Нужно запомнить интегрирующий множитель $\mu = e^{-\int\!p}$, а также свертывание в производную произведения.
	\end{Note}

	\part{Уравнения, не разрешимые относительно произодной}
	\section[Разрешимые уравнения]{Уравнение, разрешимое относительно производной}
	\begin{Ex}
		Уравнение $\hr{y'}^3 -2yx = 0$ очевидно является разрешимым относительно производной:
		$y' = \sqrt[3]{2yx}$.
	\end{Ex}
	\begin{Ex}\label{ex_4}
		Рассмотрим уравнение $(y'-2x)(y'+2x) = 0$.
		
		Рассмотрим отдельно решения уравнений $y'=2x$ и $y' = -2x$. Хочется сказать, что вместе эти решения дадут общее решение исходного. Однако это не так! Очевидно, что решениями являются параболы с ветвями вниз и вверх соответственно. Тогда можно посмотреть на кривую, содержащую левую ветвь одной из парабол и правую другой (Рис. \ref{pic1}). Это также интегральная кривая, так как она очевидно гладкая.
		\begin{figure}[h!]
			\centering
			\begin{tikzpicture}
				\begin{axis}[axis x line=center, axis y line=center]
					\addplot[domain=-3:3] {x^2};
					\addplot[domain=-3:3] {-x^2};
					\addplot[domain=-3:0, blue, thick] {-x^2};
					\addplot[domain=0:3, blue, thick] {x^2};
				\end{axis}				
			\end{tikzpicture}
		\caption{}\label{pic1}
		\end{figure}
		Оказывается, что только в точке с абсциссой $x=0$ возможны такие интегральные кривые, так как иначе гладкость не соблюдается.
	\end{Ex}

	\section{Метод введения параметра}
	\begin{Def}
		Функция $f: D \to \R$ задана параметрически соотношениями $x = \varphi(t), y = \psi(t)$, где $t\in I$, если $\varphi(I) = D$ и $\forall t\in I \; f(\varphi(t))=\psi(t)$.
	\end{Def}

	\begin{Note}
		Заметим, что определение можно переписать немного по-другому.
		Функция $f: D \to \R$ задана параметрически соотношениями $x = \varphi(t), y = \psi(t)$, где $t\in I$, если $\varphi(I) = D$ и множество $\hs{\hr{\varphi(t), \psi(t)} : t \in I}$ является графиком функции $f$.
		
		Нетрудно понять, что эти определения эквивалентны.
	\end{Note}

	\begin{Ex}
		Зададим функцию $f(x) = 1, x\in\hs{-1,1}$ параметрически.
		
		Например, $\case{x =\cos t,\\ y=1} \; t\in \R$. Очевидно, что это задание удовлетворяет определению.
	\end{Ex}
	\begin{Prop}
		Рассмотрим уравнение $F(x,y') = 0$ от двух переменных $x$ и $y'$. Пусть оно задает некоторую кривую $\gamma = \hc{\hr{x,y} | F(x,y') = 0}$ плоскости $xOy'$.
		
		Возьмем функцию $\varphi$ такую, что эта кривая является графиком функции $\varphi'$. 
		
		Тогда $F(x, \varphi'(x)) \equiv 0$. 
		
		Идея нахождения решения такого уравнения (в котором нельзя выразить $y'$) заключается в том, чтобы задать функцию $\gamma$ параметрически и найти $y$ также параметрически.
		
		Пусть $\varphi \in C^1\hr{\alpha,\beta}$, $\varphi' \neq 0$, $\psi\in C\hr{\alpha, \beta}$, причем эти функции задают параметрически наше уравнение $F(\varphi(t), \psi(t)) \equiv 0$.
		
		Тогда функция, задаваемая параметрически 
		\[
		\begin{cases}
			x = \varphi(t),\\
			y = g(t) = \int \psi(t)\varphi'(t)dt + c,
		\end{cases}
		\quad t \in \hr{\alpha, \beta}
		\]
		является решением уравнения $F(x, y')=0$.
		
		\begin{Note}
			Проверим, что функция действительно является решением.
			
			Во-первых, проверим, что такое задание действительно является функцией, то есть каждому $x$ соответствует ровно один $y$. Так как $\varphi'\neq 0$, то $\varphi$ строго возрастает и тогда $\varphi$ -- биекция. Рассматривая обратную $t=\varphi^{-1} (x)$, получаем $y=g \circ \varphi^{-1}$.  
			
			Во-вторых, проверим непрерывность и дифференцируемость решения. Так как $g \in C^{1}\hr{\alpha, \beta}$ и $\varphi^{-1} \in C^{1}\hr{\varphi(\alpha),\varphi(\beta)}$, то их композиция также непрерывно дифференцируема.
			
			В-третьих, проверим, что функция обращает наше уравнение в тождество.
			\[
			\left. F(x,y') \right|_{y=g\circ\varphi^{-1}(x)} = F\hr{x, \hr{g\circ \varphi^{-1}}'(x)}.
			\]
			Так как $\hr{g\circ \varphi^{-1}}'(x) = g'(\varphi^{-1}(x))\hr{\varphi^{-1}}'(x) = \hr{\psi(t)\varphi'(t)}_{t=\varphi'(x)} \cdot \frac{1}{\varphi'\hr{\varphi^{-1}(x)}} = \psi(\varphi^{-1}(x))$, то получаем
			\[
			F\hr{x, \hr{g\circ \varphi^{-1}}'(x)} = F\hr{x, \psi(\varphi^{-1}(x))} = F(\varphi(t), \psi(t)) \equiv 0.
			\]
			Таким образом, наша функция действительно действительно обращает выражение в нуль. А значит, эта функция является решением уравнения $F(x,y')~=~0$.
		\end{Note}
	\end{Prop}

	\begin{Ex}
		\[e^{y'} + y' = x.\]
		\begin{Solution}
			Параметризуем множество, задаваемое этим уравнением \[\hc{\hr{x,y'} :\; e^{y'} + y' = x}.\]
			
			Пусть $y'=t$. Тогда $x = e^t + t$.
			\begin{figure}[h!]
				\centering
				\begin{tikzpicture}
					\begin{axis}[axis x line=center, axis y line=center]
						\addplot[blue,no marks, domain=-5:5, samples=500] {e^x+x};
					\end{axis}
				\end{tikzpicture}
			\caption{}
			\end{figure}
		\begin{Note}[Основное соотношение метода введение параметра]
			\[
			dy=y_x'dx.
			\]
		\end{Note}
		Подставляя наши функции получаем
		\[
		dy = t \cdot d(e^t+t),
		\]
		\[
		dy = t \cdot \hr{e^t+1},
		\]
		\[
		y = \int t\cdot \hr{e^t+1} dt + c.
		\]
		В итоге мы пришли к той же самой формуле, что и доказали ранее. А значит подстановки подходят.
		
		Тогда следующая функция является решением
		\[
		\begin{cases}
			x= e^t+t,\\
			y = te^t-e^t+\frac12 t^2 + c
		\end{cases}
		\quad c\in \R, t \in \R.
		\]
		\end{Solution}
	\end{Ex}
	
	\begin{Prop}[Общий случай метода введения параметра]
		Пусть есть уравнение $F(x,y,y') = 0$, задающее какую-то поверхность $\sigma=\hr{\hr{x,y,y'}|F\hr{x,y,y'} = 0}$.
		
		Пусть $\begin{cases}
			x = \varphi(u,v),\\
			y=\psi(u,v),\\
			y' = \chi(u,v)
		\end{cases}$ -- параметризация $\sigma$.
	
		Подставим эту параметризацию в основное соотношение $dy = y_x' dx$.
		\[
		\psi_u' du + \psi_v' dv = \chi(u,v) \hr{\varphi_u'du + \varphi_v' dv}.
		\]
		Пусть $v=g(u, C)$ -- решение этого уравнения.
		
		Тогда получаем $\begin{cases}
			x = \varphi(u,v=g(u, C)),\\
			y=\psi(u,v=g(u, C))
		\end{cases}$ -- параметризация решений исходного уравнения.
	\end{Prop}
	\begin{Ex}
		\[xy'-y-\frac{y'}{2}\ln\frac{y'}{2}=0.\]
		\begin{Solution}
			Введем параметризацию 
			\[
			\begin{cases}
				x = u, \\
				y' = v, \\
				y = uv-\frac{v}{2}\ln{v}.
			\end{cases}
			\]
			Тогда подставляя в основное соотношение, получаем
			\[
			\cancel{vdu} + \hr{u-\frac12\ln\frac{v}{2} - \frac12}dv = \cancel{vdu},
			\]
			\[
			\hr{u-\frac12\ln\frac{v}{2} - \frac12}dv = 0,
			\]
			
			При $\hr{u-\frac12\ln\frac{v}{2} - \frac12} = 0$, получаем $v=2e^{2u-1}$. Тогда $y=e^{2x-1}$.
			
			При $dv=0$, получаем $y=cx-\frac{c}{2} \ln\frac{c}{2}$. 
			
			\begin{Note}
				Важно помнить, что это могут быть не все решения уравнения!!! Такой случай уже был рассмотрен в примере \ref{ex_4}.
			\end{Note}
		\end{Solution}
	\end{Ex}
	\section[Задача Коши]{Задача Коши для уравнения, не разрешенного относительно производной}
	\begin{Note}
		Мы уже говорили о решении задачи Коши для нормального уравнения. Однако в силу того, что уравнение $F\hr{x,y,y'} = 0$ задает не одно поле направлений, а целую совокупность, оказывается, что через одну точку могут проходить несколько интегральных кривых, однако под разными углами. Именно поэтому постановка задачи Коши для уравнения, не разрешенного относительно $y'$, требует дополнительного начального условия на $y'$.
	\end{Note}
	\begin{Def}
		Задачей Коши для уравнения $F(x,y,y') = 0$ называется задача нахождения его решения, удовлетворяющего условиям $\begin{cases}
			y(x_0) = y_0,\\
			y'(x_0) = y'_0.
		\end{cases}
	$. (Под $y'_0$ понимается какое-то числовое значение, а не производная. Такое обозначение используется для визуального соответствия.) 
	\end{Def}
	\begin{Prop}
		Чтобы задача Коши могла иметь решение, начальные данные должны быть согласованы, то есть $F(x_0, y_0, y'_0) = 0$.
	\end{Prop}
	\begin{Th}[Существование и единственность решения ЗК для уравнения, не разрешенного относительно производной]
		Пусть $F \in C^1\hr{G}$, где $G \subset \R^3_{x,y,y'}$ -- область. Пусть также точка $(x_0, y_0, y'_0) \in G$ такая, что $F(x_0,y_0, y'_0) =0$, $F_y'(x_0, y_0, y'_0) \neq 0$. Тогда в некоторой окрестности точки $x_0$ существует единственное решение задачи $F(x,y,y') = 0$ при условиях $\begin{cases}
			y(x_0) = y_0,\\
			y'(x_0) = y'_0.
		\end{cases}$
		\begin{Proof}
			TODO: PROOF
		\end{Proof}
	\end{Th}

	\begin{Def}
		Решение $\varphi$ уравнения $F(x, y,y') = 0$ на $\ha{a,b}$ называется особым, если для любой точки $x_0 \in \ha{a,b}$ найдется решение $\psi$ такое, что 
	\end{Def}
\end{document}













