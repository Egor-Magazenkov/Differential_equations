\documentclass[a5paper, 11pt]{article}

% Текст
\usepackage[utf8]{inputenc} % UTF-8 кодировка
\usepackage[russian]{babel} % Русский язык
\usepackage{indentfirst} % красная строка в первом параграфе в главе
% Отображение страниц
\usepackage{geometry} % размеры листа и отступов
\geometry{
	left=12mm,
	top=25mm,
	right=15mm,
	bottom=17mm,
	marginparsep=0mm,
	marginparwidth=0mm,
	headheight=10mm,
	headsep=7mm,
	nofoot}
\usepackage{afterpage,fancyhdr} % настройка колонтитулов
\pagestyle{fancy}
\fancypagestyle{style}{ % создание нового стиля style
	\fancyhf{} % очистка колонтитулов
	\fancyhead[LO, RE]{Дифференциальные уравнения} % название документа наверху
	\fancyhead[RO, LE]{\leftmark} % название section наверху
	\fancyfoot[RO, LE]{\thepage} % номер страницы справа внизу на нечетных и слева внизу на четных
	\renewcommand{\headrulewidth}{0.25pt} % толщина линии сверху
	\renewcommand{\footrulewidth}{0pt} % толцина линии снизу
}
\fancypagestyle{plain}{ % создание нового стиля plain -- полностью пустого
	\fancyhf{}
	\renewcommand{\headrulewidth}{0pt}
}
\fancypagestyle{title}{ % создание нового стиля title -- для титульной страницы
	\fancyhf{}
	\fancyhead[C]{{\footnotesize
			Министерство образования и науки Российской Федерации\\
			Федеральное государственное автономное образовательное учреждение высшего образования
	}}
	\fancyfoot[C]{{\large 
			Санкт-Петербург, 2022-2023
	}}
	\renewcommand{\headrulewidth}{0pt}
}

% Математика
\usepackage{amsmath, amsfonts, amssymb, amsthm} % Набор пакетов для математических текстов
\usepackage{dmvnbase} % мехматовский пакет latex-сокращений
\usepackage{cancel} % зачеркивание для сокращений
% Рисунки и фигуры
\usepackage[pdftex]{graphicx} % вставка рисунков
\usepackage{wrapfig, subfigure} % вставка фигур, обтекая текст
\usepackage{caption} % для настройки подписей
\captionsetup{figurewithin=none,labelsep=period, font={small,it}} % настройка подписей к рисункам
% Рисование
\usepackage{tikz} % рисование
\usepackage{pgfplots} % графики
\usepgfplotslibrary{polar}
% Таблицы
\usepackage{multirow} % объединение строк
\usepackage{multicol} % объединение столбцов
% Остальное
\usepackage[unicode, pdftex]{hyperref} % гиперссылки
\usepackage{enumitem} % нормальное оформление списков
\setlist{itemsep=0.15cm,topsep=0.15cm,parsep=1pt} % настройки списков
% Теоремы, леммы, определения...
\theoremstyle{definition}
\newtheorem{Def}{Определение}
\newtheorem*{Axiom}{Аксиома}
\theoremstyle{plain}
\newtheorem{Th}{Теорема}
\newtheorem{Lem}{Лемма}
\newtheorem{Cor}{Следствие}
\newtheorem*{Prop}{Предложение}
\newtheorem{Ex}{Пример}
\theoremstyle{remark}
\newtheorem*{Note}{Замечание}
\newtheorem*{Solution}{Решение}
\newtheorem*{Proof}{Доказательство}
% Свои команды
\newcommand{\comb}[1]{\left[\hspace{-4pt}\begin{array}{l}#1\end{array}\right.\hspace{-5pt} } % совокупность уравнений
% Титульный лист
\newcommand*{\titlePage}{
	\thispagestyle{title}
	\begingroup
	\begin{center}
		%		{\footnotesize
			%			Министерство образования и науки Российской Федерации\\
			%			Федеральное государственное автономное образовательное учреждение высшего образования
			%		}
		%		
		\vspace*{6ex}
		
		{\small
			САНКТ-ПЕТЕРБУРГСКИЙ НАЦИОНАЛЬНЫЙ ИССЛЕДОВАТЕЛЬСКИЙ УНИВЕРСИТЕТ ИНФОРМАЦИОННЫХ ТЕХНОЛОГИЙ, МЕХАНИКИ И ОПТИКИ	
		}
		
		\vspace*{2ex}
		
		{\normalsize
			Факультет систем управления и робототехники
		}
		
		\vspace*{15ex}
		
		{\Large \bfseries 
			Дифференциальные уравнения
		}
	\end{center}
	\vspace*{20ex}
	\begin{flushright}
		{\large 
			\underline{Выполнил}: студент гр. \textbf{R32353}\\
			\begin{flushright}
				\textbf{Магазенков Е. Н.}\\
			\end{flushright}
		}
		
		\vspace*{5ex}
		
		{\large 
			\underline{Преподаватель}: \textit{Бабушкин М. В.}
		}
	\end{flushright}	
	\newpage
	\setcounter{page}{1}
	\endgroup}



\begin{document}
	\titlePage
	\pagestyle{style}	
	\section[23.10.2022]{Линейные уравнения 1-ого порядка (23.10)}
	\begin{Ex}
		\[
		y' = y+x.
		\]
	\begin{Solution}
		Решим однородное уравнение
		\[
		y' = y.
		\]
		\[
		y = C\cdot e^x
		\]
		 Пусть $C$ -- функция, подставим решение в исходное уравнение
		 \[
		 C'(x) \cdot e^x + \cancel{C(x) \cdot e^x} = \cancel{C(x) \cdot e^x} + x
		 \]
		 \[
		 C'(x) = \frac{x}{e^x}
		 \]
		 \[
		 C(x) = \int x{e^{-x}} dx
		 \]
		 \[
		 u = x \implies u'=1, v' = e^{-x} \implies v = -e^{-x}
		 \]
		 \[
		 C(x) = -xe^{-x} - e^{-x} + c = -e^{-x}(x+1) + c
		 \]
		 So 
		 \[
		 y = (-e^{-x}(x+1) + c)e^x
		 \]
		 \[
		 y = -(x+1) + ce^x
		 \]
	\end{Solution}
	\end{Ex}
	
	\begin{Ex}
		\[
		y'+2y=y^2e^x
		\]
		\begin{Solution}
			Разделим на $y^2$
			\[
			\frac{y'}{y^2}  = -\frac{2}{y} + e^x
			\]
			Выполним замену $z = \frac{1}{y}$, тогда $z' = -\frac{y'}{y^2}$
			Тогда, подставляя в исходное уравнение, получаем
			\[
			z' = 2z  - e^x
			\]
			Решением данного линейного уравнения является функция 
			\[
			z = (e^{-x} + c) \cdot e^{2x}
			\]
			Тогда, возвращаясь к исходной переменной
			\[
			y = \frac{e^{-2x}}{e^{-x} + c}
			\]
		\end{Solution}
	\end{Ex}
	\begin{Ex}
		\[
		y' = y^2 -2e^xy+e^{2x}+e^x
		\]
		\begin{Solution}
			Уравнение имеет вид уравнения Рикатти.
			
			Очевидно, что $y=-e^x$ -- одно из решений. Тогда сделаем замену $z = y - e^x$.
			\[
			z' + e^x = (z+e^x)^2 - 2e^x(z+e^x) + e^{2x}+e^x,
			\]
			\[
			z' + e^x = z^2 + 2e^x z + e^{2x}- 2e^x z - 2e^{2x} + e^{2x}+e^x, 
			\]
			\[
			z' = z^2,
			\]
			\[
			-\frac1z = x + c,
			\]
			\[
			z = \frac{-1}{x+c}.
			\]
			Тогда, возвращаясь к исходным переменным, получаем $y=e^x - \frac{1}{x+c}$.
		\end{Solution}
	\end{Ex}

	\section[30.09.2022]{Уравнения в полных дифференциалах}
	\begin{Ex}
		\[
		(2-9xy^2)xdx + (4y^2-6x^3)ydy = 0.
		\]
		\begin{Solution}
			Рассмотрим производные коэффициентов
			\[
			(2-9xy^2)x_y' = -18x^2y,
			\]
			\[
			(4y^2-6x^3)y_x' = -18x^2y.
			\]
			Так как они совпадают, то по признаку это уравнение является уравнением в полных дифференциалах.
			
			Рассмотрим потенциал в какой-то точке с фиксированной ординатой $y_0$:
			\[
			u_x'(x,y_0) = 2x-9x^2y_0^2 \implies u(x, y_0) = x^2-3 x^3 y_0^2 +c(y_0) .
			\]
			Подставляя в уравнение $u_y' = (4y^2-6x^3)y$, получаем
			\[
			\hr{x^2-3 x^3 y^2 + c(y_0)}_y' = (4y^2-6x^3)y,
			\]
			\[
			-6x^3y + c'(y) = (4y^2-6x^3)y,
			\]
			\[
			c'(y) = 4y^3,
			\]
			\[
			c(y) = y^4+c.
			\]
			Таким образом, получаем функцию $u=x^2-3 x^3 y_0^2 + y^4+c$. Тогда уравнение $x^2-3 x^3 y^2 + y^4 = c$ неявно задает решение нашего уравнения.
		\end{Solution}
	\end{Ex}
	
	\begin{Note}[Свойства дифференциала]			
		\hspace{0pt}
		\begin{enumerate}
			\item $
			d(\alpha u+\beta v) = \alpha du _ \beta dv,
			$
			\item $
			du = d_x u + d_y u,
			$
			\item $
			d_x u = d_x \hr{u+\varphi(y)},
			$
			\item $
			\varphi(x,y)dx = d_x \hr{\int\varphi(x,y)dx}.
			$
		\end{enumerate}
	\end{Note}
	\begin{Note}
		Можно решать УПД, используя свойства дифференциала. Рассмотрим пример из лекции и попробуем решить его, упращая выражение по свойствам.
	\end{Note}
	\begin{Ex}
		\[
		e^{-y}dx - \hr{xe^{-y} + 2y} dy = 0.
		\]
		\begin{Solution}
		Раскроем скобки
		\[
		e^{-y}dx - xe^{-y} dy - 2y dy = 0.
		\]
		Пользуясь четвертым свойством, перейдем к 
		\[
		d_x(xe^{-y}) + d_y(xe^{-y}) - dy^2 = 0.
		\]
		Используя второе свойство, получим
		\[
		d(xe^{-y})-dy^2 = 0.
		\]
		По первому свойству
		\[
		d(xe^{-y}-y^2) = 0.
		\]
		Тогда общее решение будем неявно задаваться уравнением 
		\[
		xe^{-y}-y^2 = c.
		\]
	\end{Solution}
	\end{Ex}
	\begin{Ex}
		\[
		2xydx+(x^2-y^2)dy=0.
		\]
		\begin{Solution}
			\[
			2xydx+x^2dy-y^2dy=0,
			\]
			\[
			d_x\hr{x^2y} + d_y\hr{x^2y} - \frac13 dy^3 = 0,
			\]
			\[
			d\hr{x^2y-\frac13 y^3} = 0.
			\]
			Тогда $x^2y-\frac13 y^3 = c$ неявно задает решение.
		\end{Solution}
	\end{Ex}
	\begin{Ex}
		\[
			\hr{\frac{x}{\sqrt{x^2-y^2}}-1}dx - \frac{ydy}{\sqrt{x^2-y^2}} = 0.
		\]
		\begin{Solution}
		\[
			\frac{x}{\sqrt{x^2-y^2}}dx-dx - \frac{ydy}{\sqrt{x^2-y^2}} = 0,
		\]
		\[
		d_x\hr{\sqrt{x^2-y^2}} - dx + d_y\hr{\sqrt{x^2-y^2}} = 0, 
		\]
		\[
		d \sqrt{x^2-y^2} - dx = 0,
		\]
		\[
		d\hr{\sqrt{x^2-y^2}-x} = 0.
		\]
		Тогда $\sqrt{x^2-y^2}-x = c$ неявно задает решение.
		\end{Solution}
	\end{Ex}
	\begin{Ex}
		Найти интегральный множитель
		\[
		\hr{1-x^2y}dx + x^2(y-x)dy = 0.
		\]
		\begin{Solution}
			Запишем уравнение для нахождения интегрирующего множителя
			\[
			\mu_y' 	\hr{1-x^2y} - \mu x^2 = \mu_x'x^2(y-x)  + \mu \hr{2xy-3x^2}.
			\]
			Будем искать частное решение в виде $\mu = \mu(x)$.
			\[
			\mu_x'x^2(x-y) = 2\mu \hr{xy-x^2},
			\]
			\[
			\mu_x' x = -2 \mu,
			\]
			\[
			\mu = -\frac1{x^2}.
			\]
		\end{Solution}
	\end{Ex}
	
	\section[07.10.2022]{Метод введения параметра}
	\begin{Ex}
		\[
		y=\hr{y'}^2e^{y'}.
		\]
		\begin{Solution}
			Введем параметризацию
			\[
			\begin{cases}
				y'=v,\\
				y = v^2 e^v.
			\end{cases}
			\]
			Подставим в основное соотношение $dy = y_x' dx$
			\[
			(2ve^v + v^2 e^v)dv = v dx,
			\]
			\[
			(2+v)e^vdv = dx,
			\]
			\[
			x = e^v + ve^v + c.
			\]
			Таким образом, ответом является $\begin{cases}
				x = e^v + ve^v + c,\\
				y = v^2 e^v.
			\end{cases}$
		\end{Solution}
	\end{Ex}

	\begin{Ex}
		\[
		x^3 - (y')^3 = xy'.
		\]
		\begin{Solution}
			\begin{Note}
				Если кривая имеет уравнения вида $P(x,y) + Q(x,y) = 0$, где $P,Q$ -- однородные функции разной степени, то положив $y=xt$, получаем $x^{\alpha} P(1,t) + x^\beta Q(1,t) = 0$, откуда можно выразить $x$.
			\end{Note}
			Используя предложенное в замечании, введем параметризацию $y' = xt$.
			\[
			x^3 - (xt)^3 = x \cdot xt,
			\]
			\[
			x = \frac{t}{1-t^3},
			\]
			\[
			y'= \frac{t^2}{1-t^3}.
			\]
			Отсюда, используя основное соотношение $dy = y_x' dx$, получаем
			\[
			dy = \frac{t^2}{1-t^3} \frac{1-t^3 + 3t^3}{(1-t^3)^2}dt,
			\]
			\[
			dy = \frac{t^2(1+2t^3)}{(1-t^3)^3}dt,
			\]
			\[
			\begin{split}
				u = 1+2t^3 &\implies u' = 6t^2\\
				v' = \frac{t^2}{(1-t^3)^3} &\implies v = \frac{1}{6(1-t^3)^2}
 			\end{split}
			\]
			\[
			y = \frac{1+2t^3}{6(1-t^3)^2} - \frac{1}{3(1-t^3)} + c,
			\]
			\[
			y= \frac{4t^3-1}{6(1-t^3)^2}+c.
			\]
		\end{Solution}
	\end{Ex}
	
	\begin{Ex}
		\[
		y=2xy'-(y')^2.
		\]
		\begin{Solution}
			Введем параметризацию 
			\[
			\begin{cases}
				x = u,\\
				y'=v,\\
				y=2uv-v^2.			
			\end{cases}
			\]
			Отсюда, используя основное соотношение $dy = y_x' dx$, получаем
			\[
				2vdu + 2udv - 2vdv = v du,
			\]
			\[
				vdu = 2(v-u)dv,
			\]
			\[
				u' = 2-\frac{2u}{v},
			\]
			\[
				u= \hr{\frac{2v^3}{3} + c}v^{-2},
			\]
			Тогда решением является
			\[
				\begin{cases}
				x=\hr{\frac{2v^3}{3} + c}v^{-2},\\
				y = 2\hr{\frac{2v^3}{3} + c}v^{-1} - v^2.
				\end{cases}
			\]
		\end{Solution}
	\end{Ex}
	\section[14.10.2022]{Решение уравнений, не разрешенных относительно производной}
	\begin{Ex}
		\[
		y'^2 + 2x^3y' - 4x^2y = 0.
		\]
		\begin{Solution}
			Разложим левую часть на множители
			\[
			\hr{y' + x^3 + x \sqrt{x^4+4y}}\hr{y'+ x^3 - x \sqrt{x^4+4y}} = 0.
			\]
			Рассмотрим варианты:
			\begin{enumerate}
				\item $y' + x^3 + x \sqrt{x^4+4y}=0$.
				Заменой $z=x^4+4y \implies z' = 4x^3+4y'$ сведем это уравнение к уравнению 
				\[
					\frac{z'-4x^3}{4} + x^3 + x\sqrt{z} = 0,
				\]
				\[
				\frac{dz}{2\sqrt{z}} = -2xdx,
				\]
				\[
				\sqrt{z}= -x^2 + c, 
				\]
				\[
				z= (-x^2+c)^2, \quad \text{где } x^2\leqslant c.
				\]
				Тогда $y = \frac{(-x^2+c)^2-x^4}{4} = \frac{-2cx^2 + c^2}{4}$ при $x^2\leqslant c$.
				\item $y' + x^3 - x \sqrt{x^4+4y}=0$.
				
				Аналогично $y = \frac{2cx^2 + c^2}{4}$ при $x^2\geqslant -c$.
			\end{enumerate}
		
			Попробуем объединить эти решения, переопределив константы. В первом случае возьмем $A=-\frac{c}{2}$, а во втором -- $A = \frac{c}{2}$. 
			\begin{enumerate}
				\item $y=Ax^2+A^2$ при $A<0$ и $x\in \hs{-\sqrt{-2A}, \sqrt{-2A}}$,
				\item $y=Ax^2+A^2$ при $A\geqslant 0$ и $x\in\R$ и при $A<0$ и $x\in \hrs{-\infty, -\sqrt{-2A}} \cup \hsr{\sqrt{-2A}, +\infty}$.
			\end{enumerate}
		
			Видно, что эти решения объединяются в одно:
			\[
			y=Ax^2+A^2, \text{ при } A\in \R \text{ и } x \in \R.
			\]
			
			Также стоит не забывать случай $z=0$, который образуется при решении уравнений (мы делим на $\sqrt{z}$). В этом случае $y=-\frac{x^4}{4}$.
			
			Найдем дискриминантную кривую для данного уравнения.
			\[
			\begin{cases}
				F(x,y,y') = 0,\\
				F_{y'}'(x,y,y')=0,
			\end{cases}
			\Leftrightarrow
			\begin{cases}
					y'^2 + 2x^3y' - 4x^2y = 0,\\
					2y'+2x^3=0,
			\end{cases}
		\Leftrightarrow
		\]
		\[
			\Leftrightarrow
			\begin{cases}
				x^6 -2x^6-4x^2y=0,\\
				y' = -x^3.
			\end{cases}
			\Leftrightarrow
			\begin{cases}
				x^2\hr{x^4+4y} = 0,\\
				y' = -x^3.
			\end{cases}
		\]
		
		Таким образом, кривая $\mathcal{D} =\hc{(x,y) :\; x\hr{x^4+4y} = 0}$ является дискриминантной.
		
		Очевидно, что только решение $y=-\frac14 x^4$ проходит через $\mathcal{D}$. То есть это решение подозрительное на особое.
		
		Проверим по определению особого решения. Попробуем найти константу $A$ такую, что $\psi (x) = Ax^2+A^2$ удовлетворяет следующим условиям для любой точки $x_0 \in \R$
		\[
		\begin{cases}
			\psi(x_0) = -\frac14 x_0^4,\\
			\psi'(x_0) = -x_0^3,\\
			\psi(x) \not \equiv -\frac14 x^4, \quad \forall \; x \in U(x_0).
		\end{cases}
		\]
		\[
		\begin{cases}
			Ax_0^2+A^2 = -\frac14 x_0^4,\\
			2Ax_0 = -x_0^3,\\
			Ax^2+A^2 \not \equiv -\frac14 x^4 , \quad \forall \; x \in U(x_0).
		\end{cases}
		\Rightarrow
			A = -\frac12 x_0^2.
		\]
		Понятно, что при этом $A$ все условия выполняются, а значит кривая $y=-\frac14 x^4$ является особым решением исходного уравнения.
		
		\begin{figure}[h!]
			\centering
			\begin{tikzpicture}
				\begin{axis}[axis x line=center, axis y line=center]
					\foreach \t in {-1, -2,...,-7} {
						\addplot[domain=-5:5, thick, red] {\t*x^2+(\t)^2};
					}
					\addplot[domain=-5:5, thick, blue, samples=500] {-1/4*x^4};
				\end{axis}
			\end{tikzpicture}
		\end{figure}
		\end{Solution}
	\end{Ex}
	\section[11.11.2022]{Системы дифференциальных уравнений}
	\subsection{Метод последовательного интегрирования}
	\begin{Ex}
		\[
		\begin{cases}
			x'=x,\\
			y'=y.
		\end{cases}
		\]
		\begin{Solution}
			Заметим, что каждое из уравнений содержит только одну функцию, поэтому можно решить их по отдельности
			\[
			x = C_1e^t, \quad y=C_2 e^t, \quad C_1, C_2 \in \R.
			\]
		\end{Solution}
	\end{Ex}
	\begin{Ex}
		\[
		\begin{cases}
			x'=x,\\
			y'=y+x.
		\end{cases}
		\]
		\begin{Solution}
			Первое уравнение содержит только одну функцию и мы можем его решить и подставить решение во второе.
			\[
			\begin{cases}
				x=C_1 e^t,\\
				y'=y+C_1e^t
			\end{cases}
			\Leftrightarrow
			\begin{cases}
				x=C_1 e^t,\\
				y = C_2e^t+C_1te^t.
			\end{cases}
			\]
		\end{Solution}
	\end{Ex}
	\begin{Ex}
		Считая $t>0$, решить систему
		\[
		\begin{cases}
			x'=\frac{2t}{1+t^2}x,\\
			y'=-\frac1t y +x + t,
		\end{cases}
		\]
		\begin{Solution}
			\[
			\begin{split}
			&\begin{cases}
				x'=\frac{2t}{1+t^2}x,\\
				y'=-\frac1t y +x + t,
			\end{cases}
		\hspace{-0.5cm}\Leftrightarrow
			\begin{cases}
				x = C_1(1+t^2),\\
				y'=-\frac1t y +C_1(1+t^2) + t,
			\end{cases}
		\hspace{-0.5cm}\Leftrightarrow\\
		\Leftrightarrow&
		\begin{cases}
			x = C_1(1+t^2),\\
			y= (C_2 + \int (C_1(1+t^2)+t) t dt)\frac1t,
		\end{cases}
	\hspace{-0.5cm}\Leftrightarrow
	\begin{cases}
		x = C_1(1+t^2),\\
		y= (C_2 + C_1(\frac{t^2}{2} + \frac{t^4}{4}) + \frac{t^3}{3})\frac1t,
	\end{cases}
\hspace{-0.5cm}\Leftrightarrow\\
\Leftrightarrow&
	\begin{cases}
	x = C_1(1+t^2),\\
	y= C_2\frac1t + C_1(\frac{t}{2} + \frac{t^3}{4}) + \frac{t^2}{3},
\end{cases}
	\end{split}
			\]
		\end{Solution}
	\end{Ex}
	
	\subsection{Метод исключения}
	\begin{Note}
		Под исключением понимается избавление от всех функций кроме одной. 
		
		Идея заключается в дифференцировании нескольких из уравнений системы.
		
		Пусть есть система
		\[
		\begin{cases}
			x' = f(t,x,y),\\
			y' = g(t,x,y).
		\end{cases}
		\]
		При условии, что $f$ непрерывно дифференцируемая, у $x$ существует вторая производная.
		Дифференцируя первое уравнение, получаем
		\[
		x'' = \frac{df}{dt} = f'_t + f'x \cdot x' + f'_y \cdot y' = f'_t + f'_x \cdot f(t,x,y) + f'_y \cdot g(t,x,y).
		\]
		
		При условии, что $y$ выражается из первого уравнения --- $y = \beta(t,x,x')$, подставим в нашу вторую производную
		\[
		x'' = \gamma(x,t,t').
		\]
		
		Тогда если найдется решение $x(t)$, то можно подставить обратно и получить $y = \beta(t,x,x')$
	\end{Note}	

	\begin{Ex}
		\[
		\begin{cases}
			x' = 1-\frac1y,\\
			y' = \frac1{x-t}.
		\end{cases}
		\]
		\begin{Solution}
			Так как $\frac{1}{x-t}$ -- непрерывно дифференцируема, то из первого уравнения системы, получим 
			\[
			y'' = -\frac{x'-1}{(x-t)^2}  = \frac{-\frac1{y}}{(x-t)^2} 
			\]
			И подставим  второе
			\[
			y'' = -\frac{y'^2}{y},
			\]
			Однородное, решается заменой 
			\[
			y = C_2 \sqrt{C_1+2x},
			\]
			Тогда \[
			x = \frac{1}{y'} + t = \frac{1}{-\frac{1}{C_1+t}} + t = -{C_1+1} + t
			\]
		\end{Solution}
	\end{Ex}
	\subsection[18.11.2022]{Первый интеграл}
	\begin{Def}
		Функция $u: \Sigma \subset \R^n_r \to \R$ называется первым интегралом системы $r' = f(r)$, если для любого решения $\varphi$ этой системы $u(\varphi(t)) \equiv const$. 
	\end{Def}
	\begin{Note}
		Важно, что в правой части системы явно нет $t$ (такие системы называются автономными).
	\end{Note}
	\begin{Note}
		Рассмотрим задачу движения тела над поверхностью земли.
		\begin{figure}[h!]
			\centering
			\begin{tikzpicture}
				\draw[->, gray] (0,0) -- (0,3) node[above]{$x$};
				\draw (0,0) -- (3,0);
				\filldraw[blue] (1,2) circle (2pt) node[above left]{$m=1$};
				\draw[<->] (3,0) --node[right]{$x$} (3,2);
				\draw[dotted] (1,2) -- (3,2);
			\end{tikzpicture}
		\end{figure}
		По второму закону Ньютона 
		\[
		x'' = -g.
		\]
	
		Запишем эквивалентную систему данному уравнению:
		\[
		\begin{cases}
			x' = v,\\
			v' = -g.
		\end{cases}
		\]
		Ее решение есть $\begin{cases}
			x = -\frac{gt^2}{2} + c_1t + c_2,\\
			v = -gt + c_1,
		\end{cases}$
		
		Рассмотрим функцию полной механической энергии системы: 
		\[
		u(x,v) = \frac{v^2}{2} + gx.
		\]
		Докажем, что это первый интеграл системы. Подставим решение в эту функцию
		\[
		\begin{split}
		u(-\frac{gt^2}{2} + c_1t + c_2, -gt + c_1) = \frac{(-gt + c_1)^2}{2} + g\hr{-\frac{gt^2}{2} + c_1t + c_2} = \\
		= \cancel{\frac{g^2t^2}{2}} - \cancel{gtc_1} + \frac{c_1^2}{2} - \cancel{\frac{g^2t^2}{2}} + \cancel{gtc_1} + c_2g = \frac{c_1^2}{2} + c_2g.
		\end{split}
		\] 
		Таким образом, получили константу при каждом конкретном решении, а значит это первый интеграл по определению.
		
		\begin{Note}
			Можно было по-другому проверить. Например логично, что если это первый интеграл, то $u'(\varphi(t)) = 0$.
		\end{Note}
	\end{Note}
	\begin{Prop}
		С другой стороны, нелогично находить первый интеграл, уже зная решение системы. Хочется применять его в обратную сторону: используя первый интеграл, находить решение системы уравнения.
		
		Попробуем это сделать в нашем примере.
		
		Так как $u(x,v) = \frac{v^2}{2} + gx = const$, то можно выразить какую-нибудь из переменных через вторую. Тогда, подставляя в уравнение системы, мы получим уже дифференциальное уравнение относительно только одной переменной, которое решать уже легче.
		
		Например, попробуем выразить $x = \frac1g \hr{const-\frac{v^2}{2}}$ и подставим в первое уравнение системы
		\[
		x' = v \Leftrightarrow -\frac1g v v' = v \Leftrightarrow v' = -g.
		\] 
	\end{Prop}

	\begin{Prop}[Поиск первого интеграла для системы 2-ого порядка]
		Рассмотрим систему
		\[
		\begin{cases}
			x' = f(x,y),\\
			y' = g(x,y).
		\end{cases}
		\]
		Пусть $\hr{x,y}$ -- какое-то решение этой системы, тогда
		\[
		\begin{cases}
			\frac{dx}{dt} = f(x,y),\\
			\frac{dy}{dt} = g(x,y)
		\end{cases}
		\implies g(x,y) dx = f(x,y) dy.
		\]
		То есть $\hr{x,y}$ -- параметрическое решение уравнения $g(x,y) dx = f(x,y) dy$. Иначе говоря,  $\hr{x,y}$ --  параметрическое задание интегральной кривой.
		
		Пусть общее решение этого уравнения определено формулой $u(x,y)=c$. А тогда, подставляя исходное решение в это решение, получаем тождество $u(x(t),y(t)) \equiv c$. Таким образом, $u$ -- первый интеграл. 	
	\end{Prop}

	\begin{Ex}
		Решить систему уравнений при $x,y>0$
		\[
		\begin{cases}
			x' =\frac{x}{(x+y)^2},\\
			y' = \frac{y}{(x+y)^2}.
		\end{cases}
		\]
		\begin{Solution}
			\[
			\begin{cases}
				\frac{dx}{dt}= \frac{x}{(x+y)^2},\\
				\frac{dy}{dt} =  \frac{y}{(x+y)^2},
			\end{cases}
		\implies
			 \frac{y}{(x+y)^2} dx = \frac{x}{(x+y)^2} dy \implies ydx = xdy \implies y = cx.
			\]
			Таким образом, $u = \frac{y}{x}$ --  первый интеграл ($u=c$).
			
			Подставим во второе уравнение системы
			\[
			\begin{split}
			y' = \cancel{c}x' = \frac{\cancel{c}\cancel{x}}{(c+1)^2x^{\cancel{2}}} \Leftrightarrow xdx = \frac{dt}{(c+1)^2} \Leftrightarrow \frac{x^2}{2} =\\
			= \frac{t}{(c+1)^2} + C \Leftrightarrow x = \sqrt{\frac{2t}{(c+1)^2}+C}.
		\end{split}
			\]
			\[
			y = cx = c\cdot \sqrt{\frac{2t}{(c+1)^2}+C}.
			\]
		\end{Solution}
	\end{Ex}
	\begin{Ex}
			Решить систему уравнений при $x>z>0$ и $y>0$
			\[
			\begin{cases}
				x' = x^2 + z^2,\\
				y' = y(x-z),\\
				z' = 2xz.
			\end{cases}
			\]
			\begin{Solution}
				Заметим, что первое и третье уравнения системы не зависят от $y$. Решим их, как систему двух уравнений, используя первый интеграл
				\[
				2xzdx = (x^2+z^2)dz \implies \frac{dx}{dz} = \frac{x^2+z^2}{2xz} = \frac{1}{2z} x + \frac{z}{2}x^{-1},
				\]
				Это уравнение Бернулли, замена $t=\frac{x^2}{2}$.
				\[
				xx' = \frac{1}{2z} x^2 + \frac{z}{2} \implies t' = \frac{1}{z} t + \frac{z}{2} \implies t = \hr{C + \int\frac{z}{2} \cdot e^{-\ln z} } \cdot e^{\ln z}
				\]
			\end{Solution}
	\end{Ex}
	\section[25.11.2022]{Использование теоремы Пикара для нахождения приближенного решения задачи Коши}
	\begin{Ex}
		Укажите какой-нибудь промежуток, на котором существует единственное решение при $|t|<1$, $|x|<1$.
		\[
		x' = t^2+x^2, \qquad x(0) = 0.
		\]
		\begin{Proof}
			Понятно, что заданное множество $G:$ $|t|<1$, $|x|<1$ -- это квадрат, который является областью. А также $f(t,x) = t^2+x^2 \in C(G)$. 
			
			Рассмотрим производную функции $f$ по всем переменным кроме~$t$
			\[
			f'_x = 2x \in C(G).
			\]
			Таким образом область и функция удовлетворяют теореме Пикара с простым условием.
			
			Рассмотрим прямоугольник $\Pi$ со сторонами $a=\frac12$ и $b=\frac12$. 
			\begin{Note}
				На самом деле не всегда следует искать $\|f\|$. Достаточно просто ограничить его каким-то $M$: $\|f\| \leqslant M$. Тогда очевидно, что $h\geqslant \min\hc{a,\frac{b}{M}}$. А значит, если на отрезке $\hs{0,h}$ существует единственное решение, то на $\hs{0,\min\hc{a,\frac{b}{M}}}$ тоже.
			\end{Note}
		
			Тогда, так как $\|f\| \leqslant |t|^2 + |x|^2 \leqslant \hr{\frac12}^2 + \hr{\frac12}^2 = \frac12$, то на отрезке $\hs{-\frac12, \frac12}$ по теореме Пикара существует единственное решение.
		\end{Proof}
	\end{Ex}

	\begin{Ex}
		Укажите какой-нибудь промежуток, на котором существует единственное решение при $|t|<1$
		\[
		x' = t+\sin(t^2+x), \qquad x(0)=0.
		\]
		\begin{Solution}
			Заданное условием $|t|<1$ множество $G$ является областью, при этом $f(t,x) =  t+\sin(t^2+x) \in C(G)$. Рассмотрим $f'_x(t,x) = \cos(t^2+x) \in C(G)$. 
			
			Рассмотрим прямоугольник со сторонами $a=\frac12$, $b=\frac32$. Тогда $\|f\| = |t+\sin(t^2+x)| \leqslant |t| + |\sin(t^2+x)| \leqslant \frac12 + 1 = \frac32$.
			
			А значит по теореме Пикара с простым условием, получаем, что для $h = \min \hc{\frac12, \frac{\frac32}{\frac32}} = \frac12$ на отрезке $\hs{-h, h}$ существует единственное решение задачи Коши.
			
			\begin{Note}
				На самом деле, взяв $a=1-\varepsilon$, $b=(2-\varepsilon)(1-\varepsilon)$, где $\varepsilon \to 0$, получим, что на отрезке вида $\hs{-(1-\varepsilon), 1-\varepsilon}$ существует единственное решение. 
			\end{Note}
		\end{Solution}
	\end{Ex}
	\begin{Prop}
		В общем случае последовательность из доказательства теоремы Пикара образуется
		\[
		\begin{split}
		&\varphi_0(t) = r_0,\\
		&\varphi_m(t) = r_0 + \int\limits_{t_0}^t f(\tau, \varphi_{m-1}(\tau))d\tau.
		\end{split}
		\]
	\end{Prop}
	\begin{Ex}
		Построить третье приближение Пикара $x_3$
		\[
		x'=t-x^2, \qquad x(0)=0.
		\]
		\begin{Solution}
			\[
			\varphi_0(t) = 0,
			\]
			\[
			\varphi_1(t) = \int\limits_{0}^t \tau d\tau = \frac{t^2}{2}, 
			\]
			\[
			\varphi_2(t) = \int\limits_{0}^{t} (\tau - \hr{\frac{\tau^2}{2}}^2) d\tau = \frac{t^2}{2} - \frac{t^5}{20},
			\]
			\[
			\varphi_3(t) = \int\limits_{0}^{t} \hr{\tau - \hr{\frac{\tau^2}{2} - \frac{\tau^5}{20}}^2} = \frac{t^2}{2} - \frac{t^5}{20} + \frac{t^8}{160} - \frac{t^{11}}{4400}.
			\]
		\end{Solution}
	\end{Ex}
	\begin{Note}
		\[
		\int\varphi(t)dt = \begin{bmatrix}
			\int\varphi_1(t) dt, \\
			\int\varphi_2(t) dt
		\end{bmatrix}.
		\]
	\end{Note}
	\begin{Note}
		Для применения теоремы Пикара к уравнениям высшего порядка, необходимо переходить к эквивалентной системе уравнений первого порядка.
	\end{Note}
	\begin{Ex}
		Построить второе приближение Пикара
		\[
		y'' + (y')^2 - 2y = 0, \qquad y(0) = 1, y'(0) = 0.
		\]
		\begin{Proof}
			Запишем эквивалентную систему дифференциальных уравнений: пусть $y_1 = y, y_2 = y'$
			\[
			\begin{cases}
				y_1' = y_2,\\
				y_2' = -y_2^2 +2y_1.
			\end{cases}
			\]
			Тогда 
			\[
			\varphi_0 = \rbmat{1\\0},
			\]
			\[
			\varphi_1 = \rbmat{1\\0} + \int\limits_0^t \rbmat{0\\2\cdot 1-0^2}d\tau = \rbmat{1\\2t},
			\]
			\[
			\varphi_2 = \rbmat{1\\0} + \int\limits_0^t \rbmat{2\tau\\2\cdot1 - 4\tau^2} d\tau = \rbmat{t^2+1\\2t-\frac43 t^3}.
			\]
			Тогда приближение Пикара для исходного уравнения есть $y=y_1 = t^2+1$.
		\end{Proof}
	\end{Ex}
	\section[09.12.2022]{Продолжение решений}
	\begin{Ex}
		Доказать, что решение задачи Коши 
		\[
		y' = x^3-y^3, \qquad y(x_0) = y_0
		\]
		продолжимо на $\hrs{x_0, +\infty}$. 
		\begin{Proof}
			Доказательство основывается на рассмотрении всевозможных квадратов - компактов с центром в начале, и утверждении, что интегральная кривая выходит по правой границе квадрата.  из области $x\geqslant y$ возрастающая ($y'=(x-y)(x^2+xy+y^2)\geqslant 0$) кривая не может попасть в область $y\geqslant x$ (интересное доказательство через  определение дифференцируемости и сравнение со значением $y$). а из $y\geqslant x$ она точно перейдет в $x\geqslant y$. тогда получается мы точно выйдем по правой границе.
		\end{Proof} 
	\end{Ex}
	\section[23.12.2022]{Линейные однородные системы}
	\begin{Ex}
		\[
		\begin{cases}
			x' = -x-2y,\\
			y'=3x+4y.
		\end{cases}
		\]
		\begin{Solution}
			Запишем систему в матричном виде $r' = A r$
			\[
			\begin{bmatrix}
				x'\\y'
			\end{bmatrix}
			=
			\begin{bmatrix}
				-1 &-2\\
				3 & 4\\
			\end{bmatrix}
			\begin{bmatrix}
				x\\y
			\end{bmatrix}.
			\]
			Найдем собственные числа
			\[
			\det \hr{A-\lambda E} = 0 \Leftrightarrow (-1-\lambda)(4-\lambda) + 6 = 0 \Leftrightarrow \lambda \in \hc{1,2}.
			\]
			Найдем собственные векторы
			\begin{itemize}
				\item[$\boxed{\lambda = 1}$]
				\[
				\hr{A-E} h_1 = 0 \Leftrightarrow \begin{bmatrix}
					-2 & -3\\
					2 & 3
				\end{bmatrix}h_1 = 0 \Leftrightarrow h_1 = \begin{bmatrix}
					\alpha\\-\alpha.
				\end{bmatrix}
				\]
				\item[$\boxed{\lambda = 2}$]
				\[
				\hr{A-2E} u_1 = 0 \Leftrightarrow \begin{bmatrix}
					-3 & -2\\
					3 & 2
				\end{bmatrix}u_1 = 0 \Leftrightarrow u_1 = \begin{bmatrix}
			\beta\\
			-\frac32 \beta
		\end{bmatrix}.
				\]
			\end{itemize}
		Тогда фундаментальным решением будет 
		\[
		\begin{bmatrix}
			e^{t}\\
			e^{t}\\
		\end{bmatrix},
		\begin{bmatrix}
			2e^{2t}\\
			-3e^{2t}
		\end{bmatrix}
		\]
		\end{Solution}
	\end{Ex}
	\begin{Ex}
		\[
		\begin{cases}
		x' = 2x-y, \\
		y'=x+2y.
	\end{cases}
		\]
		\begin{Solution}
			В матричном виде
			\[
			\begin{bmatrix}
				x'\\y'
			\end{bmatrix} = 
		\begin{bmatrix}
			2 & -1\\
			1 & 2
		\end{bmatrix}
	\begin{bmatrix}
		x\\ y
	\end{bmatrix}.
			\]
			Найдем собственные числа 
			\[
			\begin{vmatrix}
				2-\lambda & -1\\
				1 & 2-\lambda
			\end{vmatrix} = 0 \Leftrightarrow 
			4-4\lambda + \lambda^2 + 1 = 0 \Leftrightarrow \lambda = 2 \pm i.
			\]
			Найдем собственные векторы
			\begin{itemize}
				\item[$\boxed{\lambda=2+i}$] 
				\[
				\begin{bmatrix}
				-i & -1\\
				1 & -i
				\end{bmatrix}h_1 = 0 \Leftrightarrow h_1 = \begin{bmatrix}
				\lambda\\
				-i\lambda 
				\end{bmatrix}
				\]			
							\item[$\boxed{\lambda=2-i}$] 
				\[
				\begin{bmatrix}
				i & -1\\
				1 & i
				\end{bmatrix}u_1 = 0 \Leftrightarrow u_1 = \begin{bmatrix}
				\beta\\
				i\beta
				\end{bmatrix}
				\]			
			\end{itemize}
			Тогда овеществленным решением будет 
				\[
				\Re{e^{(2+i)t}
			\begin{bmatrix}
				1\\
				-i\\
			\end{bmatrix}},
		\Im{e^{(2+i)t}
			\begin{bmatrix}
				1\\
				-i\\
		\end{bmatrix}},
			\]
			то есть
				\[
			e^{2t}
				\begin{bmatrix}
					\cos t\\
					\sin t\\
			\end{bmatrix},
			e^{2t}
				\begin{bmatrix}
					\sin t\\
					-\cos t\\
			\end{bmatrix}.
			\]
			
		\end{Solution}
	\end{Ex}
	\begin{Ex}
		\[
		\begin{cases}
			x' = -4x+2y+5z,\\
			y' = 6x-y-6z,\\
			z' = -8x + 3y + 9z.
		\end{cases}
		\]
		\begin{Solution}
			Собственные числа: $\lambda_1 = 2$, $\lambda_2 = 1$.
			Найдем собственные и присоединенные векторы
			\begin{itemize}
				\item[$\boxed{\lambda_1 = 2}$]
				\[
				\begin{bmatrix}
					-6 & 2 & 5\\
					6 & -3 & -6\\
					-8 & 3 & 7
				\end{bmatrix} h_1 = 0 \Leftrightarrow \begin{bmatrix}
				-6 & 2 & 5\\
				0 & -1 & -1\\
				0 & 2 & 2 
			\end{bmatrix} h_1 = 0 \Leftrightarrow h_1 = \begin{bmatrix}
			 \frac12\lambda \\
			-\lambda \\
			\lambda 
		\end{bmatrix}.
				\]
				\item[$\boxed{\lambda_2 = 1}$]
			\[
			\begin{bmatrix}
				-5 & 2 & 5\\
				6 & -2 & -6\\
				-8 & 3 & 8
			\end{bmatrix} u_1 = 0 \Leftrightarrow \begin{bmatrix}
				-5 & 2 & 5\\
				0 & 2 & 0\\
				0 & 1 & 0 
			\end{bmatrix} u_1 = 0 \Leftrightarrow u_1 = \begin{bmatrix}
				\lambda\\
				0 \\
				\lambda
			\end{bmatrix}.
			\]
			Найдем присоединенный к нему
			\[
			\begin{bmatrix}
				-5 & 2 & 5\\
				6 & -2 & -6\\
				-8 & 3 & 8
			\end{bmatrix} u_2 = \begin{bmatrix}
			1\\0\\1
		\end{bmatrix} \Leftrightarrow 
			\begin{bmatrix}
				-5 & 0 & 5\\
				0 & 2 & 0\\
				0 & 1 & 0 
			\end{bmatrix} u_2 = \begin{bmatrix}
			-5 \\ 6 \\ 3
		\end{bmatrix},
			\]
			\[
			u_2 = \begin{bmatrix}
				\gamma +1 \\3\\\gamma
			\end{bmatrix}.
			\]
			\end{itemize}
		Тогда фундаментальное решение есть
		\[
		r = C_1 e^{2t} \begin{bmatrix}
			\frac12\\-1\\1
		\end{bmatrix} + C_2 e^{t} \begin{bmatrix}
		1\\0\\1
	\end{bmatrix} + C_3 t e^t \begin{bmatrix}
	1\\3\\0
\end{bmatrix}.
		\]
		\end{Solution}
	\end{Ex}

	\section[17.02.2023]{Линейные неоднородные системы}
	\begin{Ex}
		Решить систему методом вариации постоянных
		\[
		\begin{cases}
			x' = -2x -4y+1+4t, \\
			y' = -x + y + \frac32 t^2\\
		\end{cases}
		\]
		\begin{Solution}
			Рассмотрим соответствующую однородную систему 
			\[
			r' = \underbrace{\begin{bmatrix}
				-2 & -4 \\
				-1 & 1
			\end{bmatrix}}_A r.
			\]
			
			Найдем собственные числа матрицы
			\[
			\begin{vmatrix}
				-2-\lambda & -4 \\
				-1 & 1-\lambda 
			\end{vmatrix} = \lambda^2 +\lambda-6 = 0,
			\]
			Собственные числа $\lambda_1= -3$, $\lambda_2 = 2$.
			
			Найдем собственные векторы
			\begin{itemize}
				\item[$\lambda_1 = -3$] \[
				\begin{bmatrix}
					1 & -4 \\
					-1 & 4 
				\end{bmatrix} \equiv \begin{bmatrix}
				1 & -4 \\
				0 & 0
			\end{bmatrix}.
			\]
			То есть $h_1 = \hr{4\alpha, \alpha}^T$. Возьмем $h_1 = \hr{4,1}^T$.
			
				\item[$\lambda_1 = 2$] \[
			\begin{bmatrix}
				-4 & -4 \\
				-1 & -1 
			\end{bmatrix} \equiv \begin{bmatrix}
				1 & 1\\
				0 & 0
			\end{bmatrix}.
			\]
			То есть $h_2 = \hr{\alpha, -\alpha}^T$. Возьмем $h_2 = \hr{1,-1}^T$.
			\end{itemize}
		
			Тогда фундаментальная матрица получается 
			\[
			\Phi = \begin{bmatrix}
				4e^{-3t} & e^{2t} \\
				e^{-3t} & -e^{2t} \\
			\end{bmatrix}.
			\]
			
			
			А тогда общее решение исходной системы имеет вид 
			\[
			r=  \Phi \cdot C, \quad \text{где } \Phi \cdot C' = \begin{bmatrix}
				1+4t\\\frac32 t^2
			\end{bmatrix}.
			\]
			Найдем $C$
			\[
			\begin{cases}
				4e^{-3t}C_1'+  e^{2t}C_2' = 1+4t ,\\
				e^{-3t}C_1' - e^{2t}C_2' = \frac32 t^2.
			\end{cases}
		\Leftrightarrow
			\begin{cases}
				C_1' = \frac15 e^{3t} (1+4t+\frac32 t^2),\\
				C_2' = \frac15 e^{-2t}  (1+4t-6t^2)
			\end{cases}
			\]
			\[
			\begin{cases}
			C_1 = \frac1{10} e^{3t} t^2 + \frac15 e^{3t} t + A_1, \\
			C_2 = \frac35 e^{-2t} t^2 + \frac15 e^{-2t} t + A_2
			\end{cases}
			\]
			
			Таким образом, получаем общее решение исходной системы
			\[
			\begin{split}
			r = \Phi \cdot C &= \begin{bmatrix}
				4e^{-3t} & e^{2t} \\
				e^{-3t} & -e^{2t} \\
			\end{bmatrix} \cdot \begin{bmatrix}
			\frac1{10} e^{3t} t^2 + \frac15 e^{3t} t + A_1 \\
			\frac35 e^{-2t} t^2 + \frac15 e^{-2t} t + A_2
		\end{bmatrix} = \\
	& =\begin{bmatrix}
		\frac25t^2 + \frac45 t + 4e^{-3t}A_1 + \frac3{5} t^2 + \frac15 t + e^{2t} A_2\\
		\frac1{10} t^2 + \frac15 t + e^{-3t}A_1 - \frac35 t^2 -\frac15 t - e^{-2t} A_2
	\end{bmatrix}=\\
			&= \begin{bmatrix}
				t^2 + t+ 4e^{-3t}A_1 + e^{2t}A_2\\
				-\frac12t^2 + e^{-3t}A_1- e^{-2t}A_2
			\end{bmatrix}.
		\end{split}
			\]
		\end{Solution}
	\end{Ex}

	\begin{Ex}
		Решить задачу Коши, используя матричную экспоненту
		\[
		\begin{cases}
			x' = x+4y,\\
			y' = -x + 5y,
		\end{cases} \qquad r(0) = \hr{1~1}^T.
		\]
		\begin{Solution}
			Знаем, что $r=e^{A \cdot \hr{t-0}}\begin{bmatrix}
		1\\1
	\end{bmatrix}$ является решением системы.
	
	Разложим в Жорданову форму матрицу $A$. Собственное число $\lambda_{1,2} = 3$.
	\[
	 	h_1 = \hr{2,1}^T
	\]
	\[
		h_2 = \hr{1,1}^T
	\]
	
	 То есть матрица перехода $T = \begin{bmatrix}
	 	2 & 1 \\
	 	1 & 1
	 \end{bmatrix}$, Жорданова матрица $J = \begin{bmatrix}
	 3 & 1 \\ 0 & 3
 \end{bmatrix}$.

Найдем обратную к матрице перехода
	\[
	T^{-1} = \begin{bmatrix}
		1 & -1 \\
		-1 & 2
	\end{bmatrix}.
	\]
	
	Таким образом, можно записать по формуле для матричной экспоненты
	\[
	e^{At} = T \cdot e^{3t}\cdot T^{-1} = \begin{bmatrix}
		2 & 1 \\
		1 & 1
	\end{bmatrix}  \begin{bmatrix}
	1 & t \\ 0 & 1
\end{bmatrix} \begin{bmatrix}
1 & -1 \\
-1 & 2
\end{bmatrix}e^{3t} = \begin{bmatrix}
-2t+1 & 4t \\ -t & 2t+1
\end{bmatrix}e^{3t}.
	\]
	Тогда решением системы является 
	\[
	r = \begin{bmatrix}
		-2t+1 & 4t \\ -t & 2t+1
	\end{bmatrix}e^{3t} \cdot \begin{bmatrix}
	1\\1
\end{bmatrix} = \begin{bmatrix}
2t+1\\t+1
\end{bmatrix} e^{3t}.
	\]
		\end{Solution}
	\end{Ex}

	\begin{Ex}
		Вычислить матричную экспоненту для матрицы
		\[
		A = \begin{bmatrix}
			0 & 1 & 0\\
			0 & 0 & 0 \\
			0 & 0 & 2
		\end{bmatrix}.
		\]
		\begin{Solution}
			Заметим, что $A^n = 2^n \cdot \begin{bmatrix}
				0 & 0 & 0 \\
				0 & 0 & 0 \\
				0 & 0 & 1
			\end{bmatrix}$.
		
		Тогда по определению
		\[
		e^A = \sum\limits_{n=0}^\infty \frac{A^n}{n!} = I + A + \begin{bmatrix}
			0 & 0 & 0 \\
			0 & 0 & 0 \\
			0 & 0 & 1
		\end{bmatrix} \underbrace{\sum\limits_{n=2}^\infty \frac{2^n}{n!}}_{e^2-1-2} = \begin{bmatrix}
		1 & 1 & 0\\
		0& 1 & 0\\
		0 & 0& e^2
	\end{bmatrix}.
		\]
		\end{Solution}
	\end{Ex}

	\begin{Ex}
		Найти определитель матричной экспоненты, не вычисляя саму матричную экспоненту
		\[
		A = \begin{bmatrix}
			1 & 0 & 3\\
			-1 & 2 & 0\\
			0 & 1 & -1
		\end{bmatrix}.
		\]
		\begin{Solution}
			Заметим, что $\det e^{At} = W(t)$.
			
			По формуле Остроградского-Лиувилля 
			\[
			\det e^{At}  = \det e^{A\cdot 0} e^{\int\limits_0^t \tr A d\tau} = e^{2t}.
			\]
			А тогда, подставляя $t=1$, получаем искомое.
			\[
			\det e^{A} = e^2.
			\]
		\end{Solution}
	\end{Ex}

	\section[03.03.2023]{Линейные уравнения}
	\begin{Ex}
		\[
		4y''-8y'+5y=0.
		\]
		\begin{Solution}
			Корни характеристического уравнения $\lambda = 1 \pm \frac12 i$.
			
			Тогда решения есть $e^{\hr{1 + \frac12 i}t}, e^{\hr{1 - \frac12 i}t}$. Овеществляя их, рассматривая вещественную и мнимую часть, получаем
			\[
			e^{t}\cos \frac t2, e^t \sin \frac t2.
			\]
			Тогда общее решение есть $y = C_1e^{t}\cos \frac t2 + C_2 e^t \sin \frac t2$.
		\end{Solution}
	\end{Ex}
	\begin{Ex}
		\[
		y'' - 2y' + y = \frac{e^t}{t}, \quad t>0.
		\]
		\begin{Solution}
			Решение однородного есть $y = C_1 e^t + C_2 t e^t$.
			
			Тогда найдем $C_i$ из системы
			
			\[
			\begin{bmatrix}
				\Lambda e^t, \Lambda te^t
			\end{bmatrix}
			\begin{bmatrix}
				C_1'\\C_2'
			\end{bmatrix}
			=
			\begin{bmatrix}
				0\\\frac{e^t}{t}
			\end{bmatrix}.
			\]
			\[
			\begin{cases}
				C_1'e^t + C_2'te^t=0,\\
				C_1' e^t + C_2' \hr{e^t+te^t} = \frac{e^t}{t}
			\end{cases}
			\]
			\[
			\begin{cases}
				C_1' = -1,\\
				C_2' = \frac{1}{t}
			\end{cases}
			\]
			\[
			\begin{cases}
				C_1 = -t + C_3\\
				C_2 = \ln t + C_4
			\end{cases}
			\]
			
			Таким образом, общее решение есть 
			\[
			y = \hr{-t + C_3} e^t + \hr{\ln t + C_4} t e^t
			\]
		\end{Solution}
	\end{Ex}

	\begin{Def}
		$t^{k} e^{\lambda t}$ называется квазиодночленом. Соответственно, $\sum\limits_{j=1}^n t^{k_j} e^{\lambda_j t}$ называется квазимногочленом.
	\end{Def}

	\begin{Lem}
		Если $q(t) = e^{\gamma t} p_k(t)$, где $p_k$ -- многочлен степени $R$, то уравнение $Ly=q(t)$ имеет решение 
		\[
		y = t^m e^{\gamma t}r_k(t),
		\] 
		где $r_k$ -- многочлены степени $r$ (с неопределенными коэффициентами), $m=0$, если $\gamma$ -- не характеристическое число, $m$ -- кратность $\gamma$, если $\gamma$ -- характеристическое число.
	\end{Lem}

	\begin{Ex}
		\[
		y''-y = t^2-t+1.
		\]
		\begin{Solution}
			\[
			q(t) = t^2 - t+ 1 = e^{\gamma t } p_k(t), \text{ где $\gamma = 0, k=2$}.
			\]
			
			Так как характеристические корни $\lambda = \pm 1$, не равны $\gamma=0$, то $m=0$.
			
			Тогда $y = A_2 t^2 + A_1 t + A_0$ -- частное решение.
			
			Подставляя в исходное, получаем 
			\[
			2A_2 - \hr{A_2 t^2 + A_1 t + A_0} = t^2-t+1,
			\]
			По неопределенным коэффициентам находим
			\[
			\begin{cases}
				A_2 = -1,\\
				A_1 = 1,\\
				A_0 = -3.
			\end{cases}
			\]
			Таким образом, получаем, что общее решение исходного уравнения есть сумма общего решения однородного и частного
			\[
			y = C_1 e^{-t} + C_2 e^t -t^2 + t - 3 
			\]
		\end{Solution}
	\end{Ex}

	\begin{Ex}
		\[
		y'' + y = 2e^t + 1.
		\]
		\begin{Solution}
			Решим уравнение $y'' + y = 2e^t$
			\[
			q(t) = 2e^t \implies k=0, \gamma=1.
			\]
			Характеристические корни есть $y=\pm i$. Тогда $m=0$.
			
			Таким образом, частное решение есть
			\[
			y = A_0 e^t. 
			\]
			Подставляя в исходное, получаем
			\[
			A_0 e^t + A_0 e^t = 2e^t,
			\]
			\[
			A_0 = 1.
			\]
			
			Решим уравнение $y'' + y = 1$
			\[
			q(t) = 1 \implies \gamma = 0, k = 0.
			\]
			Тогда частное решение имеет вид 
			\[
			y = A = 1.
			\]
			
			А значит частное решение исходного есть сумма частных решений предыдущих
			\[
			y = e^t+1,
			\]
			Тогда общее решение, как сумма общего решение однородного и частного решения
			\[
			y = C_1 \cos t + C_2 \sin t + e^t + 1
			\]
		\end{Solution}
	\end{Ex}
	\section[10.03.2023]{Линейные уравнения с переменными коэффициентами}
	\begin{Ex}
		\[
		y''-ty=0.
		\]
		Оказывается, что решения в виде элементарных функций у такого уравнения не существует. Однако по теореме Коши решения все-таки есть и выражается формулой Тейлора.
		
		Пусть $\varphi$ -- решение и имеет вид $\varphi = \sum\limits_{k=0}^{\infty} a_k t^k$. Найдем значения коэффициентов.
		
		Для этого подставим в исходное уравнение
		\[
		\sum\limits_{k=0}^{\infty} a_k t^k - t \sum\limits_{k=0}^{\infty} a_k t^k \equiv 0,
		\]
		\[
		\sum\limits_{k=2}^{\infty} k(k-1) a_k t^{k-2} - \sum\limits_{k=0}^{\infty} a_k t^{k+1} \equiv 0,
		\]
		Сведем к одному ряду от нуля до бесконечности заменой индексов
		\[
			\sum\limits_{k=0}^{\infty} (k+2)(k+1)a_{k+2} t^k - \sum\limits_{k=1}^{\infty} a_{k-1} t^k \equiv 0,
		\]
		\[
		\sum\limits_{k=1}^{\infty} \hr{(k+2)(k+1)a_{k+2} - a_{k-1}} t^k + 2a_2 \equiv 0,
		\]
		Тогда получаем 
		\[
		\begin{cases}
			a_2 = 0,\\
			(k+2)(k+1)a_{k+2} - a_{k-1} = 0, \quad k\geqslant 1.
		\end{cases}
		\]
		
		Второе равенство на самом деле есть реккурентное задание 
		\[
		a_m = \frac{a_{m-3}}{m(m-1)} \quad m\geqslant 3 = \frac{a_{m-6}}{m(m-1)(m-3)(m-6)} = \dots = \frac{a_{m \, mod \, 3}}{m!!! (m-1)!!!}.
		\]
		Таким образом,
		\begin{itemize}
			\item если $m\,mod\, 3 = 2$, то $a_m = 0$,
			\item если $m\,mod\, 3 = 1$, то $a_m = \frac{a_{1}}{m!!! (m-1)!!!}$,
			\item если $m\,mod\, 3 = 0$, то $a_m = \frac{a_{0}}{m!!! (m-1)!!!}$.
		\end{itemize}
		И тогда 
		\[
		\varphi(t) = a_0\sum\limits_{\stackrel{m=0}{m \,mod\,3=0}}^\infty \frac{t^m}{m!!!(m-1)!!!} + a_1\sum\limits_{\stackrel{m=0}{m \,mod\,3=1}}^\infty \frac{t^m}{m!!!(m-1)!!!}.
		\]
		При этом слагаемые оказываются действительно линейно независимыми решениями, так как вронскиан в нуле равен 1, а значит это действительно общее решение. 
	\end{Ex}
	\begin{Ex}
		\[
		xy''+2y'+xy=0.
		\]
		\begin{Solution}
			Пусть $\varphi$ -- решение и имеет вид $\varphi = \sum\limits_{k=0}^{\infty} a_k x^k$. Найдем значения коэффициентов.
			
			Для этого подставим в исходное уравнение
			\[
			x\sum\limits_{k=2}^{\infty} k(k-1) a_k x^{k-2} + 2\sum\limits_{k=1}^{\infty}k  a_k x^{k-1} + x \sum\limits_{k=0}^{\infty} a_k x^k \equiv 0,
			\]
			\[
			\sum\limits_{k=2}^{\infty} k(k-1) a_k x^{k-1} + 2\sum\limits_{k=1}^{\infty}k  a_k x^{k-1}  +  \sum\limits_{k=2}^{\infty} a_{k-2} x^{k-1} \equiv 0,
			\]
			\[
			\sum\limits_{k=2}^{\infty} \hr{k(k-1) a_k + 2ka_k + a_{k-2}} + 2a_1 \equiv 0,
			\]
			\[
			\begin{cases}
				a_1 = 0,\\
				k(k-1) a_k + 2ka_k + a_{k-2} = 0, \quad k \geqslant 2
			\end{cases}
			\]
			Второе уравнение задает
			\[
			a_k = \frac{-a_{k-2}}{k^2+k}= \frac{a_{k-4}}{(k^2+k)((k-2)^2 + (k-2))} = \dots = \frac{(-1)^{[\frac{k}{2}]} a_{k\,mod\,2}}{(k+1)!}
			\]
			Таким образом, $a_{2m+1} = 0$, $a_{2m} = \frac{(-1)^{m} a_{0}}{(2m+1)!}$
			
			И решение есть
			\[
			\varphi(x) = a_0 \sum\limits_{m=0}^{\infty} \frac{(-1)^{m}}{(2m+1)!} x^{2m} = \frac{a_0}{x} \sum\limits_{m=0}^{\infty} \frac{(-1)^{m}}{(2m+1)!} x^{2m+1} = \frac{a_0}{x} \sin x.
			\]
			Однако видно, что это решение не задает линейное пространство нужной размерности, а значит не является общим.
			
			Воспользуемся теоремой Остроградского-Лиувилля уже для приведенного уравнения $y'' + \frac{2}{x} y' + y = 0$
			\[
			\begin{vmatrix}
				\varphi_1 & \varphi_2 \\
				\varphi_1' & \varphi_2' \\
			\end{vmatrix} = W(x_0) e^{-\int_{x_0}^x \frac{2}{x}dx},
			\]
			\[
			\begin{vmatrix}
				\frac{\sin x}{x} & \varphi_2 \\
				\frac{x\cos x - \sin x}{x^2} & \varphi_2' \\
			\end{vmatrix} = C e^{-\int \frac{2}{x}dx},
			\]
			\[
			\frac{\sin x}{x} \varphi_2' = \frac{x\cos x - \sin x}{x^2} \varphi_2 + \frac{1}{x^2},
			\]
			\[
			\varphi_2' = \hr{\ctg x - \frac1x} \varphi_2 + \frac{1}{x\sin x},
			\]
			\[
			\varphi_2 = \hr{C + \int \frac1{x\sin x} e^{-\int \hr{\ctg x - \frac1x}dx}} e^{\int \hr{\ctg x - \frac1x}dx},
			\]
			\[
			\varphi_2 = \hr{C + \int \frac1{x\sin x} e^{-\ln(\sin x) + \ln x}} e^{\ln(\sin x) - \ln x},
			\]
			\[
			\varphi_2 = \hr{C + \int \frac1{x\sin x} \frac{x}{\sin x}} \frac{\sin x}{x},
			\]
			\[
			\varphi_2 = \hr{C - \ctg x} \frac{\sin x}{x},
			\]
			Рассмотрим какое-нибудь решение, например, при $C=0$ (при любом $C$ это решение будет линейно независимо с полученным через ряд).
			
			Таким образом, получается, что общее решение есть 
			\[
			\varphi = C_1 \frac{\sin x}{x} + C_2 \frac{\cos x}{x}
			\]
		\end{Solution}
	\end{Ex}
	\section[17.03.2023]{Устойчивости}
	\begin{Ex}
		Исследовать на устойчивость решение $\varphi(t) = t$ уравнения $x'=1+t-x$.
		\begin{Solution}
			Рассмотрим
			\[
			s' = 1+ t -(s+\varphi) - 1-t+\varphi = -s. 
			\]
			Так как решение $s=0$ является асимптотически устойчивым для этого уравнения, то любое решение также будет асимптотически устойчивым.
		\end{Solution}
	\end{Ex}
	\begin{Ex}
		Исследовать на устойчивость решение $x=0$ уравнения $x'=\sin^2 x$.
		\begin{Solution}
			\textbf{1 способ}
			
			Решим данное дифференциальное уравнение:
			
			\[
			\begin{split}
			x = \arcctg(-t+C), \quad x \in (0,\pi),\\
			x = -\pi + \arcctg(-t+C), \quad x \in (-\pi, 0),\\
			\end{split}
			\]
			Причем $C$ определяется из начального условия $x(0) = x_0$ как $C=\ctg x_0$.
			
			Рассмотрим на положительной части. Так как арктангенс в любом случае стремится к $\pi$, то появляется предположение, что устойчивости нет.
			
			Вспомним отрицание определения устойчивости
			\[
			\exists \varepsilon: \; \forall \delta \; \exists x_0 \in B_\delta(0) \; \exists t \in \hsr{0, +\infty}: \quad \hm{x(t,x_0)} \geqslant \varepsilon.
			\]
			Тогда рассматривая $\varepsilon = \frac{\pi}2$, 
			\[
			\hm{\arcctg(-t+\ctg x_0)} \geqslant \frac{\pi}{2}, 
			\]
			так как $\arcctg$ стремится к $\pi$.
			
			\textbf{2 способ}
			
			Рассмотрим фазовое пространство в окрестности точки 0. Так как фазовые скорости всегда неотрицательны, то, выйдя из точки в окрестности 0, мы никогда не вернемся туда, то есть решение неустойчивое.
		\end{Solution}
	\end{Ex}

	\begin{Ex}
		Исследовать на устойчивость решение $x=0$ уравнения $x' = -\frac{\sin^2 x}{\cos x}$.
		\begin{Solution}
			Рассмотрим фазовое пространство в окрестности точки 0. Так как фазовые скорости всегда в этой окрестности неположительные, то, выйдя из точки в окрестности 0, мы всегда вернемся в нуль, то есть решение асимптотически устойчивое.
		\end{Solution}
	\end{Ex}
	\begin{Ex}
		Найдите точки покоя уравнения
		\[
		x'=x^3-7x^2+36
		\]
		и исследуйте их на устойчивость.
		\begin{Solution}
			Точки покоя найдем из уравнения 
			\[
			x^3 - 7x^2 + 36 = 0 \Leftrightarrow x\in \{-2,3,6\}.
			\]
			При $x=-2$ в его окрестности фазовая скорость в сторону 0 оказывается положительной, а значит точка неустойчивая.
			
			При $x=-2$ в его окрестности фазовая скорость в сторону 0 оказывается положительной, а значит точка неустойчивая.
			
			При $x=-2$ в его окрестности фазовая скорость в сторону 0 оказывается положительной, а значит точка неустойчивая.
			
			
		\end{Solution}
	\end{Ex}


	\section[14.04.2023]{Функция Четаева}
	\begin{Ex}
		Рассмотрим одномерный случай, в котором $f(x)>0$.
		Тогда функция $C(x) = x$ является функцией Четаева:
		\begin{enumerate}
			\item $C(0) = 0$,
			\item $\forall \;\rho > 0 \quad B_\rho(0) \cap \hc{r|C(r)>0} = \hr{0,\rho} \not = 0$,
			\item $\forall \; r\in \bar{B}_\rho(0) \cap \hc{r|C(r)>0} \quad \grave{C}(r) = f(x) > 0$.
		\end{enumerate}
		Таким образом, функция действительно является функцией Четаева. И значит точка 0 является точкой неустойчивого положения равновесия.
	\end{Ex}
	\begin{Ex}
		\[
		\begin{cases}
			x'=-y+x^3,\\
			y'=x+y^3.
		\end{cases}
		\]
		Доказать, что положение равновесия $(0,0)$ неустойчиво. 
		\begin{Solution}
			Будем искать функцию Четаева в виде $C(x,y) = ax^{2\alpha} + by^{2\beta}$.
				\begin{enumerate}
				\item $C(0,0) = 0$,
				\item $C(x,y)>0$ при $a,b>0$,
				\item \[\begin{split}\forall \; r\in \bar{B}_\rho(0) \cap \hc{r|C(r)>0}\\ \grave{C}(r) = \begin{pmatrix}
					2a\alpha x^{2\alpha-1} & 2b\beta y^{2\beta-1}
				\end{pmatrix} \cdot \begin{pmatrix}
				-y+x^3\\x+y^3
			\end{pmatrix} = \\
		= 2a\alpha x^{2\alpha+2} - 2a\alpha x^{2\alpha-1}y +  2b\beta y^{2\beta+2} + 2b\beta y^{2\beta-1}x = \\
		= \underbrace{2a\alpha x^{2\alpha+2} +  2b\beta y^{2\beta+2}}_{>0} + 2xy \hr{b\beta y^{2\beta-2} + a\alpha x^{2\alpha-2}}
		\end{split}\].
			Выберем теперь коэффициенты так, чтобы $b\beta y^{2\beta-2} + a\alpha x^{2\alpha-2} = 0$. Например, $\alpha = \beta = 2$, $a=-b$. Тогда получаем, что $\grave{C}(r) > 0$.
			\end{enumerate}
		\end{Solution}
	\end{Ex}

	\begin{Note}
		Вспомним теорему о том, что фазовые траектории системы $\begin{cases}
			x' = f(x,y),\\
			y' = g(x,y)
		\end{cases}$ можно построить как интегральные кривые уравнения $g(x,y)dx - f(x,y)dy=0$ в области $\hc{\hr{x,y} | f(x,y)\neq 0 \vee g(x,y)\neq 0}$.
	\end{Note}

	\begin{Ex}
		\[
		\begin{cases}
			x' = -y-2xy,\\
			y'=x+2x^2.
		\end{cases}
		\]
		Исследовать точку $(0,0)$ на устойчивость.
		\begin{Solution}
			Воспользуемся теоремой, о которой только что вспомнили и рассмотрим уравнение
			\[
			\hr{x+2x^2} dx - \hr{-y-2xy} dy = 0.
			\]
			Будем искать интегрирующий множитель $\mu = \mu(y)$
			\[
			\mu (-2y) = \mu'_y (x+2x^2),
			\]
			\[
			\mu'_y = -\mu \frac{2y}{x+2x^2},
			\]
			\[
			\mu = \frac{e^{-y^2}}{x+2x^2}.
			\]
			\[
			e^{-y^2} dx + \frac{e^{-y^2}}{x+2x^2} (y+2xy)dy = 0,
			\]
			\[
			dx + \frac{y}{x}dy = 0,
			\]
			\[
			y^2 = -x^2 + C,
			\]
			\[
			y^2+x^2 = C.
			\]
			То есть интегральные кривые данного уравнения есть окружности, а значит и фазовые траектории выглядят, как окружности. Таким образом, попав в любую точку, мы будем оставаться на окружности, то есть положение устойчиво.
		\end{Solution}
	\end{Ex}
	\begin{Note}
		Также часто для анализа может помочь переход в полярные координаты.
		\[
		\begin{cases}
			x'=f(x,y),\\
			y'=g(x,y).
		\end{cases}
		\longrightarrow \begin{cases}
			r' = u(r,\varphi),\\
			\varphi' = v(r,\varphi).
		\end{cases}
		\]
		Однако в лоб очень легко $u,v$ не получить. Поэтому воспользуемся хитростями:
		
		Вспомним, что $r^2 = x^2+y^2$. Тогда
		\[
		\frac{d r^2}{dt} = \frac{d x^2}{dt} + \frac{d y^2}{dt},
		\]
		\[
		\begin{split}
		2rr' = 2xx' + 2yy' \implies r' = \frac{xx'+yy'}{r} = \frac{xf(x,y)+yg(x,y)}{r} = \\
		= f(r\cos \varphi, r\sin \varphi) \cos\varphi + g(r\cos \varphi, r\sin \varphi) \sin \varphi.
		\end{split}
		\]
		
		Также знаем, что $\tg \varphi = \frac yx$. Тогда
		\[
		\frac{d \tg \varphi}{dt} = \frac{d \frac yx}{dt},
		\]
		\[
		\frac{1}{\cos^2 \varphi} \varphi' = \frac{y'x-x'y}{x^2},
		\]
		\[
		\begin{split}
		\varphi' &= \frac{y'x-x'y}{\hr{\frac{x}{\cos \varphi}}^2} = \frac{y'x-x'y}{r^2} =\\
		&= \frac{g(r\cos \varphi, r\sin \varphi)\cos \varphi -f(r\cos \varphi, r\sin \varphi)\sin \varphi}{r}.
		\end{split}
		\]
	\end{Note}
	\begin{Ex}
		\[
		\begin{cases}
			x' = -y-x(x^2+y^2),\\
			y' = x-y(x^2+y^2).
		\end{cases}
		\]
		Исследовать на устойчивость точку $(0,0)$.
		\begin{Solution}
			Перейдем к полярным координатам
			\[
			r' = \frac{x (-y-x(x^2+y^2)) + y (x-y(x^2+y^2))}{\sqrt{x^2+y^2}} = (-x^2-y^2)\sqrt{x^2+y^2} = -r^3,
			\]
			\[
			\varphi' = \frac{-y (-y-x(x^2+y^2)) + x (x-y(x^2+y^2))}{x^2+y^2} = 1.
			\]
			Заметим, что фазовое пространство данной системы (Рис. \ref{spiral}) есть спирали, закручивающиеся к точке $(0,0)$. То есть для любой точки пространства мы придем к началу координат. А значит есть асимптотическая устойчивость.
			\begin{figure}[h!]
				\centering
				\begin{tikzpicture}
					\begin{axis}[
						xmin = -1, xmax = 1,
						ymin = -1, ymax = 1,
						zmin = 0, zmax = 1,
						axis equal image,
						view = {0}{90},
						]
						\addplot3[
						quiver = {
							u = {-y-x*(x^2+y^2)},
							v = {x-y*(x^2+y^2)},
							scale arrows = 0.25,
						},
						-stealth,
						domain = -1:1,
						domain y = -1:1,
						] {0};
					\end{axis}
				\end{tikzpicture}
			\caption{Фазовое пространство} \label{spiral}
			\end{figure}
		\end{Solution}
	\end{Ex}

	\begin{Ex}
		\[
		\begin{cases}
			x' = -y + x(1-x^2-y^2),\\
			y' = x + y(1-x^2-y^2).
		\end{cases}
		\]
		\begin{Solution}
			Перейдем к полярным координатам
			\[
			r' = \frac{x(-y + x(1-x^2-y^2))+y(x + y(1-x^2-y^2))}{\sqrt{x^2+y^2}} = r-r^3,
			\]
			\[
			\varphi' = \frac{-y(-y + x(1-x^2-y^2))+x(x + y(1-x^2-y^2))}{x^2+y^2} = \frac{x^2+y^2}{x^2+y^2} = 1.
			\]
			\begin{tikzpicture}
				\draw[->] (0,0) -- (0,4) node[above]{$r$};
				\draw (0,0) circle (1pt) node[left]{0};
				\draw (0,2) circle (1pt) node[left]{1};
				\draw[->] (0,0) -- (0,1) node[left]{+};
				\draw[->] (0,4) -- (0,3)node[left]{-};
				
			\end{tikzpicture}
	
	
		\end{Solution}
	\end{Ex}
	
	\section[21.04.2023]{Циклы}
	\begin{Ex}
		\[
		\begin{cases}
			x' = -y + x(1-2x^2-3y^2),\\
			y' = x+y(1-2x^2-3y^2).
		\end{cases}
		\]
		Доказать, что существуют циклы между $x^2+y^2=1$ и $x^2+y^2=\frac14$.
		\begin{Solution}
			Рассмотрим окружность $x^2+y^2=1$ в параметрическом виде $\begin{cases}
				x = \cos t,\\y=\sin t.
			\end{cases}$. Тогда ее нормаль есть $n(t) = \begin{pmatrix}
			\cos t \\ \sin t
		\end{pmatrix}$.
		\[
		n(t) \cdot f = -1-\sin^2 t  < 0.
		\]
		А значит любая траектория не выйдет за окружность.
		
		Аналогично для меньшей окружности:  $n = (\frac12 \cos t, \frac12 \sin t)^T$, $n\cdot f = \frac14 (\frac34 - \frac14 \sin^2 t) > 0$.
		\end{Solution}
	\end{Ex}
	\begin{Ex}
		\[
		\begin{cases}
			x' = 2x+y - xe^{x^2+y^2},\\
			y' = -x + 2y - ye^{x^2+y^2}.
		\end{cases}
		\]
		Доказать существование цикла (найти $c_o$, $c_i$ в виде окружностей).
		\begin{Solution}
			\[
			\begin{cases}
				x(t) = \sqrt{\ln2}\cos t,\\
				y(t) = -\sqrt{\ln2}\sin t.
			\end{cases}
			\]
		\end{Solution}
	\end{Ex}

	\begin{Ex}
		\[
		x'' + p(x) x' + q(x) = 0, \quad p(x) > 0, p, q \in C^1(\R).
		\]
		Доказать, что уравнение не имеет периодических решений.
		\begin{Solution}
			Перейдем к системе уравнений, равносильной уравнению
			\[
			
			\]
		\end{Solution}
		
	\end{Ex}
\end{document}
