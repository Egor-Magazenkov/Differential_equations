\documentclass[a5paper, 11pt]{article}

% Текст
\usepackage[utf8]{inputenc} % UTF-8 кодировка
\usepackage[russian]{babel} % Русский язык
\usepackage{indentfirst} % красная строка в первом параграфе в главе
% Отображение страниц
\usepackage{geometry} % размеры листа и отступов
\geometry{
	left=12mm,
	top=25mm,
	right=15mm,
	bottom=17mm,
	marginparsep=0mm,
	marginparwidth=0mm,
	headheight=10mm,
	headsep=7mm,
	nofoot}
\usepackage{afterpage,fancyhdr} % настройка колонтитулов
\pagestyle{fancy}
\fancypagestyle{style}{ % создание нового стиля style
	\fancyhf{} % очистка колонтитулов
	\fancyhead[LO, RE]{Дифференциальные уравнения} % название документа наверху
	\fancyhead[RO, LE]{\leftmark} % название section наверху
	\fancyfoot[RO, LE]{\thepage} % номер страницы справа внизу на нечетных и слева внизу на четных
	\renewcommand{\headrulewidth}{0.25pt} % толщина линии сверху
	\renewcommand{\footrulewidth}{0pt} % толцина линии снизу
}
\fancypagestyle{plain}{ % создание нового стиля plain -- полностью пустого
	\fancyhf{}
	\renewcommand{\headrulewidth}{0pt}
}
\fancypagestyle{title}{ % создание нового стиля title -- для титульной страницы
	\fancyhf{}
	\fancyhead[C]{{\footnotesize
			Министерство образования и науки Российской Федерации\\
			Федеральное государственное автономное образовательное учреждение высшего образования
	}}
	\fancyfoot[C]{{\large 
			Санкт-Петербург, 2022-2023
	}}
	\renewcommand{\headrulewidth}{0pt}
}

% Математика
\usepackage{amsmath, amsfonts, amssymb, amsthm} % Набор пакетов для математических текстов
\usepackage{dmvnbase} % мехматовский пакет latex-сокращений
\usepackage{cancel} % зачеркивание для сокращений
% Рисунки и фигуры
\usepackage[pdftex]{graphicx} % вставка рисунков
\usepackage{wrapfig, subfigure} % вставка фигур, обтекая текст
\usepackage{caption} % для настройки подписей
\captionsetup{figurewithin=none,labelsep=period, font={small,it}} % настройка подписей к рисункам
% Рисование
\usepackage{tikz} % рисование
\usepackage{pgfplots} % графики
% Таблицы
\usepackage{multirow} % объединение строк
\usepackage{multicol} % объединение столбцов
% Остальное
\usepackage[unicode, pdftex]{hyperref} % гиперссылки
\usepackage{enumitem} % нормальное оформление списков
\setlist{itemsep=0.15cm,topsep=0.15cm,parsep=1pt} % настройки списков
% Теоремы, леммы, определения...
\theoremstyle{definition}
\newtheorem{Def}{Определение}
\newtheorem*{Axiom}{Аксиома}
\theoremstyle{plain}
\newtheorem{Th}{Теорема}
\newtheorem{Lem}{Лемма}
\newtheorem{Cor}{Следствие}
\newtheorem*{Prop}{Предложение}
\newtheorem{Ex}{Пример}
\theoremstyle{remark}
\newtheorem*{Note}{Замечание}
\newtheorem*{Solution}{Решение}
\newtheorem*{Proof}{Доказательство}
% Свои команды
\newcommand{\comb}[1]{\left[\hspace{-4pt}\begin{array}{l}#1\end{array}\right.\hspace{-5pt} } % совокупность уравнений
% Титульный лист
\newcommand*{\titlePage}{
	\thispagestyle{title}
	\begingroup
	\begin{center}
		%		{\footnotesize
			%			Министерство образования и науки Российской Федерации\\
			%			Федеральное государственное автономное образовательное учреждение высшего образования
			%		}
		%		
		\vspace*{6ex}
		
		{\small
			САНКТ-ПЕТЕРБУРГСКИЙ НАЦИОНАЛЬНЫЙ ИССЛЕДОВАТЕЛЬСКИЙ УНИВЕРСИТЕТ ИНФОРМАЦИОННЫХ ТЕХНОЛОГИЙ, МЕХАНИКИ И ОПТИКИ	
		}
		
		\vspace*{2ex}
		
		{\normalsize
			Факультет систем управления и робототехники
		}
		
		\vspace*{15ex}
		
		{\Large \bfseries 
			Дифференциальные уравнения
		}
	\end{center}
	\vspace*{20ex}
	\begin{flushright}
		{\large 
			\underline{Выполнил}: студент гр. \textbf{R32353}\\
			\begin{flushright}
				\textbf{Магазенков Е. Н.}\\
			\end{flushright}
		}
		
		\vspace*{5ex}
		
		{\large 
			\underline{Преподаватель}: \textit{Бабушкин М. В.}
		}
	\end{flushright}	
	\newpage
	\setcounter{page}{1}
	\endgroup}



\begin{document}
	\titlePage
	\pagestyle{style}	
	\section[23.10.2022]{Линейные уравнения 1-ого порядка (23.10)}
	\begin{Ex}
		\[
		y' = y+x.
		\]
	\begin{Solution}
		Решим однородное уравнение
		\[
		y' = y.
		\]
		\[
		y = C\cdot e^x
		\]
		 Пусть $C$ -- функция, подставим решение в исходное уравнение
		 \[
		 C'(x) \cdot e^x + \cancel{C(x) \cdot e^x} = \cancel{C(x) \cdot e^x} + x
		 \]
		 \[
		 C'(x) = \frac{x}{e^x}
		 \]
		 \[
		 C(x) = \int x{e^{-x}} dx
		 \]
		 \[
		 u = x \implies u'=1, v' = e^{-x} \implies v = -e^{-x}
		 \]
		 \[
		 C(x) = -xe^{-x} - e^{-x} + c = -e^{-x}(x+1) + c
		 \]
		 So 
		 \[
		 y = (-e^{-x}(x+1) + c)e^x
		 \]
		 \[
		 y = -(x+1) + ce^x
		 \]
	\end{Solution}
	\end{Ex}
	
	\begin{Ex}
		\[
		y'+2y=y^2e^x
		\]
		\begin{Solution}
			Разделим на $y^2$
			\[
			\frac{y'}{y^2}  = -\frac{2}{y} + e^x
			\]
			Выполним замену $z = \frac{1}{y}$, тогда $z' = -\frac{y'}{y^2}$
			Тогда, подставляя в исходное уравнение, получаем
			\[
			z' = 2z  - e^x
			\]
			Решением данного линейного уравнения является функция 
			\[
			z = (e^{-x} + c) \cdot e^{2x}
			\]
			Тогда, возвращаясь к исходной переменной
			\[
			y = \frac{e^{-2x}}{e^{-x} + c}
			\]
		\end{Solution}
	\end{Ex}
	\begin{Ex}
		\[
		y' = y^2 -2e^xy+e^{2x}+e^x
		\]
		\begin{Solution}
			Уравнение имеет вид уравнения Рикатти.
			
			Очевидно, что $y=-e^x$ -- одно из решений. Тогда сделаем замену $z = y - e^x$.
			\[
			z' + e^x = (z+e^x)^2 - 2e^x(z+e^x) + e^{2x}+e^x,
			\]
			\[
			z' + e^x = z^2 + 2e^x z + e^{2x}- 2e^x z - 2e^{2x} + e^{2x}+e^x, 
			\]
			\[
			z' = z^2,
			\]
			\[
			-\frac1z = x + c,
			\]
			\[
			z = \frac{-1}{x+c}.
			\]
			Тогда, возвращаясь к исходным переменным, получаем $y=e^x - \frac{1}{x+c}$.
		\end{Solution}
	\end{Ex}

	\section[30.09.2022]{Уравнения в полных дифференциалах}
	\begin{Ex}
		\[
		(2-9xy^2)xdx + (4y^2-6x^3)ydy = 0.
		\]
		\begin{Solution}
			Рассмотрим производные коэффициентов
			\[
			(2-9xy^2)x_y' = -18x^2y,
			\]
			\[
			(4y^2-6x^3)y_x' = -18x^2y.
			\]
			Так как они совпадают, то по признаку это уравнение является уравнением в полных дифференциалах.
			
			Рассмотрим потенциал в какой-то точке с фиксированной ординатой $y_0$:
			\[
			u_x'(x,y_0) = 2x-9x^2y_0^2 \implies u(x, y_0) = x^2-3 x^3 y_0^2 +c(y_0) .
			\]
			Подставляя в уравнение $u_y' = (4y^2-6x^3)y$, получаем
			\[
			\hr{x^2-3 x^3 y^2 + c(y_0)}_y' = (4y^2-6x^3)y,
			\]
			\[
			-6x^3y + c'(y) = (4y^2-6x^3)y,
			\]
			\[
			c'(y) = 4y^3,
			\]
			\[
			c(y) = y^4+c.
			\]
			Таким образом, получаем функцию $u=x^2-3 x^3 y_0^2 + y^4+c$. Тогда уравнение $x^2-3 x^3 y^2 + y^4 = c$ неявно задает решение нашего уравнения.
		\end{Solution}
	\end{Ex}
	
	\begin{Note}[Свойства дифференциала]			
		\hspace{0pt}
		\begin{enumerate}
			\item $
			d(\alpha u+\beta v) = \alpha du _ \beta dv,
			$
			\item $
			du = d_x u + d_y u,
			$
			\item $
			d_x u = d_x \hr{u+\varphi(y)},
			$
			\item $
			\varphi(x,y)dx = d_x \hr{\int\varphi(x,y)dx}.
			$
		\end{enumerate}
	\end{Note}
	\begin{Note}
		Можно решать УПД, используя свойства дифференциала. Рассмотрим пример из лекции и попробуем решить его, упращая выражение по свойствам.
	\end{Note}
	\begin{Ex}
		\[
		e^{-y}dx - \hr{xe^{-y} + 2y} dy = 0.
		\]
		\begin{Solution}
		Раскроем скобки
		\[
		e^{-y}dx - xe^{-y} dy - 2y dy = 0.
		\]
		Пользуясь четвертым свойством, перейдем к 
		\[
		d_x(xe^{-y}) + d_y(xe^{-y}) - dy^2 = 0.
		\]
		Используя второе свойство, получим
		\[
		d(xe^{-y})-dy^2 = 0.
		\]
		По первому свойству
		\[
		d(xe^{-y}-y^2) = 0.
		\]
		Тогда общее решение будем неявно задаваться уравнением 
		\[
		xe^{-y}-y^2 = c.
		\]
	\end{Solution}
	\end{Ex}
	\begin{Ex}
		\[
		2xydx+(x^2-y^2)dy=0.
		\]
		\begin{Solution}
			\[
			2xydx+x^2dy-y^2dy=0,
			\]
			\[
			d_x\hr{x^2y} + d_y\hr{x^2y} - \frac13 dy^3 = 0,
			\]
			\[
			d\hr{x^2y-\frac13 y^3} = 0.
			\]
			Тогда $x^2y-\frac13 y^3 = c$ неявно задает решение.
		\end{Solution}
	\end{Ex}
	\begin{Ex}
		\[
			\hr{\frac{x}{\sqrt{x^2-y^2}}-1}dx - \frac{ydy}{\sqrt{x^2-y^2}} = 0.
		\]
		\begin{Solution}
		\[
			\frac{x}{\sqrt{x^2-y^2}}dx-dx - \frac{ydy}{\sqrt{x^2-y^2}} = 0,
		\]
		\[
		d_x\hr{\sqrt{x^2-y^2}} - dx + d_y\hr{\sqrt{x^2-y^2}} = 0, 
		\]
		\[
		d \sqrt{x^2-y^2} - dx = 0,
		\]
		\[
		d\hr{\sqrt{x^2-y^2}-x} = 0.
		\]
		Тогда $\sqrt{x^2-y^2}-x = c$ неявно задает решение.
		\end{Solution}
	\end{Ex}
	\begin{Ex}
		Найти интегральный множитель
		\[
		\hr{1-x^2y}dx + x^2(y-x)dy = 0.
		\]
		\begin{Solution}
			Запишем уравнение для нахождения интегрирующего множителя
			\[
			\mu_y' 	\hr{1-x^2y} - \mu x^2 = \mu_x'x^2(y-x)  + \mu \hr{2xy-3x^2}.
			\]
			Будем искать частное решение в виде $\mu = \mu(x)$.
			\[
			\mu_x'x^2(x-y) = 2\mu \hr{xy-x^2},
			\]
			\[
			\mu_x' x = -2 \mu,
			\]
			\[
			\mu = -\frac1{x^2}.
			\]
		\end{Solution}
	\end{Ex}
	
	\section[07.10.2022]{Метод введения параметра}
	\begin{Ex}
		\[
		y=\hr{y'}^2e^{y'}.
		\]
		\begin{Solution}
			Введем параметризацию
			\[
			\begin{cases}
				y'=v,\\
				y = v^2 e^v.
			\end{cases}
			\]
			Подставим в основное соотношение $dy = y_x' dx$
			\[
			(2ve^v + v^2 e^v)dv = v dx,
			\]
			\[
			(2+v)e^vdv = dx,
			\]
			\[
			x = e^v + ve^v + c.
			\]
			Таким образом, ответом является $\begin{cases}
				x = e^v + ve^v + c,\\
				y = v^2 e^v.
			\end{cases}$
		\end{Solution}
	\end{Ex}

	\begin{Ex}
		\[
		x^3 - (y')^3 = xy'.
		\]
		\begin{Solution}
			\begin{Note}
				Если кривая имеет уравнения вида $P(x,y) + Q(x,y) = 0$, где $P,Q$ -- однородные функции разной степени, то положив $y=xt$, получаем $x^{\alpha} P(1,t) + x^\beta Q(1,t) = 0$, откуда можно выразить $x$.
			\end{Note}
			Используя предложенное в замечании, введем параметризацию $y' = xt$.
			\[
			x^3 - (xt)^3 = x \cdot xt,
			\]
			\[
			x = \frac{t}{1-t^3},
			\]
			\[
			y'= \frac{t^2}{1-t^3}.
			\]
			Отсюда, используя основное соотношение $dy = y_x' dx$, получаем
			\[
			dy = \frac{t^2}{1-t^3} \frac{1-t^3 + 3t^3}{(1-t^3)^2}dt,
			\]
			\[
			dy = \frac{t^2(1+2t^3)}{(1-t^3)^3}dt,
			\]
			\[
			\begin{split}
				u = 1+2t^3 &\implies u' = 6t^2\\
				v' = \frac{t^2}{(1-t^3)^3} &\implies v = \frac{1}{6(1-t^3)^2}
 			\end{split}
			\]
			\[
			y = \frac{1+2t^3}{6(1-t^3)^2} - \frac{1}{3(1-t^3)} + c,
			\]
			\[
			y= \frac{4t^3-1}{6(1-t^3)^2}+c.
			\]
		\end{Solution}
	\end{Ex}
	
	\begin{Ex}
		\[
		y=2xy'-(y')^2.
		\]
		\begin{Solution}
			Введем параметризацию 
			\[
			\begin{cases}
				x = u,\\
				y'=v,\\
				y=2uv-v^2.			
			\end{cases}
			\]
			Отсюда, используя основное соотношение $dy = y_x' dx$, получаем
			\[
				2vdu + 2udv - 2vdv = v du,
			\]
			\[
				vdu = 2(v-u)dv,
			\]
			\[
				u' = 2-\frac{2u}{v},
			\]
			\[
				u= \hr{\frac{2v^3}{3} + c}v^{-2},
			\]
			Тогда решением является
			\[
				\begin{cases}
				x=\hr{\frac{2v^3}{3} + c}v^{-2},\\
				y = 2\hr{\frac{2v^3}{3} + c}v^{-1} - v^2.
				\end{cases}
			\]
		\end{Solution}
	\end{Ex}
	\section[14.10.2022]{Решение уравнений, не разрешенных относительно производной}
	\begin{Ex}
		\[
		y'^2 + 2x^3y' - 4x^2y = 0.
		\]
		\begin{Solution}
			Разложим левую часть на множители
			\[
			\hr{y' + x^3 + x \sqrt{x^4+4y}}\hr{y'+ x^3 - x \sqrt{x^4+4y}} = 0.
			\]
			Рассмотрим варианты:
			\begin{enumerate}
				\item $y' + x^3 + x \sqrt{x^4+4y}=0$.
				Заменой $z=x^4+4y \implies z' = 4x^3+4y'$ сведем это уравнение к уравнению 
				\[
					\frac{z'-4x^3}{4} + x^3 + x\sqrt{z} = 0,
				\]
				\[
				\frac{dz}{2\sqrt{z}} = -2xdx,
				\]
				\[
				\sqrt{z}= -x^2 + c, 
				\]
				\[
				z= (-x^2+c)^2, \quad \text{где } x^2\leqslant c.
				\]
				Тогда $y = \frac{(-x^2+c)^2-x^4}{4} = \frac{-2cx^2 + c^2}{4}$ при $x^2\leqslant c$.
				\item $y' + x^3 - x \sqrt{x^4+4y}=0$.
				
				Аналогично $y = \frac{2cx^2 + c^2}{4}$ при $x^2\geqslant -c$.
			\end{enumerate}
		
			Попробуем объединить эти решения, переопределив константы. В первом случае возьмем $A=-\frac{c}{2}$, а во втором -- $A = \frac{c}{2}$. 
			\begin{enumerate}
				\item $y=Ax^2+A^2$ при $A<0$ и $x\in \hs{-\sqrt{-2A}, \sqrt{-2A}}$,
				\item $y=Ax^2+A^2$ при $A\geqslant 0$ и $x\in\R$ и при $A<0$ и $x\in \hrs{-\infty, -\sqrt{-2A}} \cup \hsr{\sqrt{-2A}, +\infty}$.
			\end{enumerate}
		
			Видно, что эти решения объединяются в одно:
			\[
			y=Ax^2+A^2, \text{ при } A\in \R \text{ и } x \in \R.
			\]
			
			Также стоит не забывать случай $z=0$, который образуется при решении уравнений (мы делим на $\sqrt{z}$). В этом случае $y=-\frac{x^4}{4}$.
			
			Найдем дискриминантную кривую для данного уравнения.
			\[
			\begin{cases}
				F(x,y,y') = 0,\\
				F_{y'}'(x,y,y')=0,
			\end{cases}
			\Leftrightarrow
			\begin{cases}
					y'^2 + 2x^3y' - 4x^2y = 0,\\
					2y'+2x^3=0,
			\end{cases}
		\Leftrightarrow
		\]
		\[
			\Leftrightarrow
			\begin{cases}
				x^6 -2x^6-4x^2y=0,\\
				y' = -x^3.
			\end{cases}
			\Leftrightarrow
			\begin{cases}
				x^2\hr{x^4+4y} = 0,\\
				y' = -x^3.
			\end{cases}
		\]
		
		Таким образом, кривая $\mathcal{D} =\hc{(x,y) :\; x\hr{x^4+4y} = 0}$ является дискриминантной.
		
		Очевидно, что только решение $y=-\frac14 x^4$ проходит через $\mathcal{D}$. То есть это решение подозрительное на особое.
		
		Проверим по определению особого решения. Попробуем найти константу $A$ такую, что $\psi (x) = Ax^2+A^2$ удовлетворяет следующим условиям для любой точки $x_0 \in \R$
		\[
		\begin{cases}
			\psi(x_0) = -\frac14 x_0^4,\\
			\psi'(x_0) = -x_0^3,\\
			\psi(x) \not \equiv -\frac14 x^4, \quad \forall \; x \in U(x_0).
		\end{cases}
		\]
		\[
		\begin{cases}
			Ax_0^2+A^2 = -\frac14 x_0^4,\\
			2Ax_0 = -x_0^3,\\
			Ax^2+A^2 \not \equiv -\frac14 x^4 , \quad \forall \; x \in U(x_0).
		\end{cases}
		\Rightarrow
			A = -\frac12 x_0^2.
		\]
		Понятно, что при этом $A$ все условия выполняются, а значит кривая $y=-\frac14 x^4$ является особым решением исходного уравнения.
		
		\begin{figure}[h!]
			\centering
			\begin{tikzpicture}
				\begin{axis}[axis x line=center, axis y line=center]
					\foreach \t in {-1, -2,...,-7} {
						\addplot[domain=-5:5, thick, red] {\t*x^2+(\t)^2};
					}
					\addplot[domain=-5:5, thick, blue, samples=500] {-1/4*x^4};
				\end{axis}
			\end{tikzpicture}
		\end{figure}
		\end{Solution}
	\end{Ex}
	\end{document}
